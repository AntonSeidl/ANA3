\subsection{Das Lebesgue-Integral}

Das Ziel ist eine größere Klasse integrierbarer Funktionen, als beim Riemannintegral. Dazu sei vorab erwähnt, dass es bei jedem plausiblen Integralsbegriff nichtintegrierbare Funktionen $f\colon [a,b] \to \mathbb{R}$ gibt.\\
\underline{Behauptung:} Es gibt keine Funktion (Maß) $\mu\colon P([0,1]) \to [0,1]$ derart, dass
\begin{itemize}
	\item $\mu([0,1]) = 1, \, \mu(\emptyset) = 0$
	\item Ist $A = A_1 \dot{\cup A_2} \dot{\cup} ... \,(A_i \subset [0,1])$, dann $\mu(A) = \sum_{i=1}^{\infty} \mu(A_i)$
	\item Für $A \subset [0,1], \alpha \in [0,1], A_{\alpha} \coloneqq \{y + \alpha | y \in A\} \subset [0,1]$ dann $\mu(A_{\alpha}) = \mu(A)$
\end{itemize}
\underline{Beweis:} Für $x,y\in [0,1]$ definiere $x \sim y \Leftrightarrow x-y \in \mathbb{Q}$\\
Das heißt $[0,1]$ zerfällt in Äquivalenzklassen und es gibt ein Repräsentantensystem $R \subset [0,1]$. $R$ enthält aus jeder Äquivalenzklasse genau ein Element.\\
Für $q\in \mathbb{Q} \cap [0,1[$ definiere $R_q = \{y + q | y \in R, y \in [0,1-q]\} \dot{\cup} \{ y+q-1|y \in ]1-q,1] \}$.\\
Dann gilt: $[0,1] = \dot{\bigcup_{q\in[0,1[}} R_q$\\
Denn die $R_q$ sind paarweise disjunkt, sonst gäbe es zwei Elemente $x\neq y$ von R mit $x-y \in \mathbb{Q}$ und für jedes $x \in [0,1]$ existiert ein $y \in R$ mit $x-y \in \mathbb{Q} \Rightarrow x = y + q \lor x = y + q - 1$ für $q\in [0,1[$ geeignet gewählt.\\
Aber: $1 = \mu([0,1]) = \sum_{q \in [0,1[} \mu(R_q) = \sum_{q \in [0,1[} \mu(R) =
\left \{
\begin{matrix}
0\\
\infty
\end{matrix}
\right . \lightning$\\
\underline{Folgerung:} Ein vernünftiger Integralbegriff für alle Funktionen $f\colon [0,1] \to \mathbb{R}$ sollte insbesondere für Indikatorfunktionen von Mengen $A \subset [0,1]$ einen Wert $\mu(A)\coloneqq \int_{a}^{b} I_A(x) \text{d}x$ liefern, mit $I_A(x) \coloneqq \left \{ \begin{matrix}
1, & x \in A\\
0, & x \ne A
\end{matrix}
\right .$ derart, dass obige Eigenschaften erfüllt sind. Also kann kein solches „universelles“ Integral existieren.

\subsubsection{Definition}
\begin{itemize}
	\item[a)] Sei $D \subseteq \mathbb{R}^n$, dann heißt $\varphi :D \to \mathbb{R}^n$ Treppenfunktion auf D, falls
		\[ \exists \text{Quader} Q_1, Q_2, ..., Q_l \subseteq \mathbb{R}^n \text{paarweise disjunkt mit} \overline{D} \supseteq \bigcup\limit_{i=1}^l \overline{Q_i}
		\text{derart, dass} \varphi \text{für jedes} i = 1,2,...,l \text{konstant auf dem Quader} Q_i \text{ist und} f(x) = 0 \forall x \in D\setminus \bigcup\limit_{i=1}^l\]
	\item[b)] Das Integral der Treppenfuktion \varphi\colon D \to \mathbb{R} aus a) ist definiert durch
		\[ \int_D \varphi(x) \text{d}x = \sum\limit_{i=1}^l \varphi(\xi_i) \mu(Q_i)\]
		wobei $\xi_i \in Q_i (i=1,...,l)$ beliebige Stützstellen sind. 
\end{itemize}

\underline{Bemerkung}
\begin{itemize}
	\item[1)] Für $n=1$ ist das äquivalent zur Definition aus AN 1/2
	\item[2)] Integraldefinition ist wohldefiniert, insbesondere unabhängig von der gewählten Unterteilung im Quader $Q_i$
	\item[3)] Menge der Treppenfuktionen auf D ist abgeschlossen unter Linearkombination
	\[(\varphi, \psi \text{Treppenfuktionen auf D} \alpha, \beta \in \mathbb{R} \Rightarrow \alpha\varphi + \beta\psi \text{Treppenfuktionen auf D})\]
	und unter Max/Min.
	\[(x \mapsto Max (\varphi(x), \psi(x)) \text{ist Treppenfunktion auf D})\]
	\item[4)] Ränder von n-dim Quadern sind n-dim Nullmengen.
\end{itemize}
Im Folgenden ist zunächst meist $D=\overline{Q}$, $Q$ ist Quader, d. h. D ist abgeschlossener Quader oder $D=\mathbb{R}^n$\\
Das ist im Folgenden definierte \underline{Lebesgue-Integral} ist im Vergleich zum Riemannintegral
\begin{itemize}
	\item allgemeiner (mehr integrierbare Funktionen)
	\item hat "schönere" (Konvergenz-) Eigenschaften
\end{itemize}

\underline{Idee:} Ähnlich wie beim Riemannintegral f durch Treppenfuktionen approximieren.
\underline{Sprechweise:} Sei $D \subseteq \mathbb{R}^n$, wir sagen, eine Eigenschaft (z. B. f stetig bei $x \in D$) ist fast überall (f.ü.) erfüllt, falls
\[N=\{x \in D | x \text{erfüllt die Eigenschaft nicht}\}\]
eine (n-dim) Nullmenge ist.

\underline{Schlüsselidee:} punktweise Konvergenz $\varphi_n \longrightarrow^{n \to \infty} f$ f. ü.\\
	(d.h. \exists N \subseteq D \text{Nullmenge:} \forall x \in D \setminus N \colon \lim_{n \to \infty} \varphi_n(x)=f(x)) und damit
	\[\int_D f(x) \text{d}x = \lim_{n \to \infty} \int_D \varphi_n(x) \text{d}x\]