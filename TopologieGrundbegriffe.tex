\subsection{Topologische Grundbegriffe}
\begin{itemize}
	\item Eine Menge $A \subset \mathbb{R}^n$ heißt \underline{beschränkt}, falls
	\[\exists_{c \in \mathbb{R}^n, r \in \mathbb{R}_{>0}} A \subset B_r(c),\] wobei \[B_r(c)\coloneqq \{x \in \mathbb{R}^n| ||x-c|| < r\}\] die offene n-dimensionale Kugel um $c$ mit Radius $r$ ist.
	\item $A \subset \mathbb{R}^n$ heißt \underline{offen} genau dann, wenn gilt:
	\[\forall_{a \in A} \exists_{\epsilon > 0} B_{\epsilon}(a) \subset A.\]
	\item $A \subset \mathbb{R}^n$ \underline{abgeschlossen} genau dann, wenn gilt:
	\[\mathbb{R} \setminus A \, \textrm{ist offen} \Leftrightarrow \, \forall_{a \in \mathbb{R}^n \setminus A} \exists_{\epsilon > 0} B_{\epsilon}(a) \cap A = \emptyset\]
	\begin{equation*}
		\begin{matrix}
		\textrm{\underline{Allgemein:} für } A \subset \mathbb{R}^n & \textrm{A ist abgeschlossen} & \Leftrightarrow & \mathbb{R}^n \setminus A \, \textrm{ist offen}\\
		& \textrm{A ist offen} & \Leftrightarrow & \mathbb{R}^n \setminus A \, \textrm{ist abgeschlossen}
		\end{matrix}\\
	\end{equation*}
	Einige Beispiele:
	\begin{itemize}
		\item[-] $]0,1] \times \,]0,1] \subset \mathbb{R}^2$ ist weder offen noch abgeschlossen.
		\item[-] Sind $U_i \subset \mathbb{R}^n (i \in I)$ offene Mengen, so ist auch $\bigcup_{i \in I} U_i$ offen im $\mathbb{R}^n$.
		\item[-] Sind $A_i \subset \mathbb{R}^n (i \in I)$ abgeschlossene Mengen, so ist auch $\bigcap_{i \in I} A_i$ abgeschlossen im $\mathbb{R}^n$.
		\item[-] Sind $U_1, U_2 \subset \mathbb{R}^n$ offen, so auch $U_1 \cap U_2$.
		\item[-] Sind $A_1, A_2 \subset \mathbb{R}^n$ abgeschlossen, so auch $A_1 \cup A_2$.
		\item[-] Seien $U_k\coloneqq B_{1+\frac{1}{k}} (0) \subset \mathbb{R}^n, k=1,2,...$ offene Bälle um die Null, dann gilt:
		\[\bigcap_{k=1}^\infty U_k = \{x \in \mathbb{R}\, |\, \Vert x\Vert \leq 1\} = \overline{B_1(0)}\]
	\end{itemize}
	\item Eine Menge $K \subset \mathbb{R}^n$ heißt \underline{kompakt} in topologischen Räumen, falls sie abgeschlossen und beschränkt ist.\\
	Für  ein kompaktes $K \subset \mathbb{R}^n$ gilt: Ist $(U_i)_{i \in I}$ eine Familie offener Mengen, das heißt $U_i \subset \mathbb{R}^n$ ist offen und gilt $K \subset \bigcup_{i \in I} U_i$, dann existiert ein $J \subset I$ mit $|J| \le \infty$ und $K \subset \bigcup_{i \in J} U_i$.\\
	
	In Worten heißt das, dass jede offene Überdeckung einer kompakten Menge eine endliche Teilmenge besitzt.
	\underline{Behauptung:}
		Sei $K \subset \mathbb{R}^n$ mit $A_i \subset K (I \in I)$ abgeschlossene Teilmengen, mit $\bigcap_{i \in I} A_i = \emptyset$, dann folgt:
		\[ \exists_{J \subset I} J > \infty \wedge \bigcap_{i \in J} A_i = \emptyset \]
	\underline{Beweis:}
		Sei $U_i \coloneqq \mathbb{R}^n \setminus A_i \, (i \in I)$, dann folgt
		\[ \bigcup_{i \in I} U_i = \bigcup_{i \in I} (\mathbb{R}^n \setminus A_i) = \mathbb{R}^n \setminus \bigcap_{i \in I} A_i = \mathbb{R}^n \setminus \emptyset = \mathbb{R}^n \supset K \]
		\[\Rightarrow \exists_{J \subset I}\colon |J| < \infty \wedge K \subset \bigcup_{i \in J} U_i\]
		\[\Rightarrow K \subset U_{i \in J} (\mathbb{R}^n \setminus A_i) = \mathbb{R}^n \setminus \bigcap_{i \in J} A_i = \emptyset\]
 \end{itemize}
