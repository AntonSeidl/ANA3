\documentclass[ngerman]{scrartcl}
\usepackage[utf8]{inputenc}
\usepackage[ngerman]{babel}
\usepackage{lmodern} 
\usepackage{graphicx}
\usepackage{amsmath}
\usepackage[amsmath,amsthm,thmmarks]{ntheorem}
\newcommand{\proofend}{\begin{flushright}$\Box$\end{flushright}}
\usepackage{amssymb}
\usepackage{mathtools}
\usepackage{tabulary}
\usepackage{booktabs}
\usepackage{enumitem}
\usepackage{stmaryrd}

\newcommand{\RM}[1]{\MakeUppercase{\romannumeral #1{.}}}
\newcommand{\integ}[2]{\int_#1 #2 \text{d}x}


\theoremstyle{definition}
\newtheorem{definition}[subsubsection]{Definition}

\theoremstyle{definition}
\newtheorem{proposition}[subsubsection]{Proposition}

\theoremstyle{definition}
\newtheorem{claim}[subsubsection]{Behauptung}
\newtheorem*{claim_n}{Behauptung}

\theoremstyle{remark}
\newtheorem*{remark}{Bemerkung}

\begin{document}
\title{Analysis $\RM{3}$}

\section*{Einführung}
Dieses Skript basiert auf einer Mitschrift der Vorlesung Analysis $\RM{3}$, des Wintersemesters 18/19 an der OTH - Regensburg, gehalten von Herrn Prof. Dr. Illies. Das allgemeine Ziel der Vorlesung ist die Integration im $\mathbb{R}^n$, genauer Gebiets-, Kurven- und Oberflächenintegrale, sowie die dazugehörenden Integrationssätze von Gauß, Stokes und Green.\\
Um dies zu erreichen werden wir das Riemannintegral im $\mathbb{R}^1$ zu dem Lebesgueintegral verallgemeinern. Dabei stellen wir fest, dass das Riemannintegral ein Spezialfall des Lebesgueintegrals ist und mit den Sätzen der majorisierten und der monotonen Konvergenz, dass zweiteres bessere Konvergenzeigenschaften hat.

\newpage

\section{Lebesgue-Integration oder Gebietsintegrale im $\mathbb{R}^n$}
\subsection{Wiederholung: Abzählbare und endliche Mengen}
\underline{Definition:}
\begin{itemize}[noitemsep]
	\item Zwei Mengen $A$ und $B$ heißen \underline{gleichmächtig}, falls eine bijektive Abbildung \mbox{$f: A\rightarrow B$} existiert.\\
	\item Eine Menge $A$ heißt \underline{abzählbar unendlich}, falls $A$ gleichmächtig zu $\mathbb{N}$ ist.\\
	\item $A$ heißt \underline{abzählbar}, falls $A$ endlich oder abzählbar unendlich ist.\\
	\item $A$ heißt \underline{überabzählbar}, falls $A$ nicht abzählbar ist.
\end{itemize}
\underline{Beispiele:}
\begin{itemize}[noitemsep]
	\item $\mathbb{N}$ ist abzählbar (unendlich)
	\item $\mathbb{Z}$ ist abzählbar (unendlich)
	\item $\mathbb{Q}$ ist abzählbar (unenedlich)
	\item Sind $A = \{a_1, a_2, ...\}$ und $B = \{b_1, b_2, ...\}$ abzählbar uneendlich, so auch \mbox{$A \cup B = \{a_1, b_1, a_2, b_2, ...\}$}
	\item Falls $A \subset B$ und $A$ überabzählbar, so ist auch $B$ überabzählbar.
	\item $a\le b \Rightarrow [a,b[$ überabzählbar\\
Beweisidee: Cantors Diagonalargument O. E. sei $a = 0, b = 1$, da für $a \leq b$ eine Bijektion $g:[a,b[ \rightarrow [0,1[$ existiert.\\
Angenommen: $[0,1[ = \{a_1, a_2, a_3, ...\}$, O. E. $\alpha_1 = 0, a_1^{(i)} = 0$\\

Seien

\begin{centering}
	\begin{tabulary}{\textwidth}{l l}
		$a_1=0,$ & $\alpha_1^{(1)}, \alpha_2^{(1)}, \alpha_3^{(1)}, ...$\\
		$a_2=0,$ & $\alpha_1^{(2)}, \alpha_2^{(2)}, \alpha_3^{(2)}, ...$\\
		$a_3=0,$ & $\alpha_1^{(3)}, \alpha_2^{(3)}, \alpha_3^{(3)}, ...$\\
		$\vdots$ &
	\end{tabulary}\\
\end{centering}
die nicht abbrechende Dezimaldarstellung der $a_j$, für $i\geq 2$, d. h. $\alpha_i^{(j)} \in \{0,1,...,9\}$ und für $j \geq 2$ sind unendlich viele der $\alpha_1^{(j)}, \alpha_2^{(j)},... \neq 0$

\begin{equation*}
	Sei \, \beta_i = \left\{
		\begin{aligned}
		1, \, & falls \, a_i^{(j)} \neq 1\\
		2, \, & falls \, a_i^{(j)} = 1
		\end{aligned}
		\right.
\end{equation*}
Betrachte die Zahl $\beta \in [0,1[$ mit Dezimaldarstellung $\beta := 0,\beta_1, \beta_2, \beta_3, ...$. Es müsste denmanch ein $i \in \mathbb{N}$ mit $\beta = a_j$ geben. Das hieße, dass für alle $i\in \mathbb{N} \, \beta_i = \alpha_i^{(j)}$ gilt, insbesondere $\beta_i = \alpha_i^{(j)}$ was ein Widerspruch zur Definition unserer $\beta_i$ ist. \proofend

\item $\mathcal{P(\mathbb{N})}$ ist gleichmächtig zu $\{(\epsilon_1, \epsilon_2, \epsilon_3, ...)| \, \epsilon_i \in \{0,1\}\} =: \{0,1\}^\mathbb{N} =: \mathcal{E}$
Es ist demnach $\mathcal{E}$ die Menge aller unendlichen 0-1-Folgen und $\mathcal{P(\mathbb{N})}$ die Potenzmenge von $\mathbb{N}$, wobei für beliebige Mengen $A$ definiert ist:
\begin{center}
	$\mathcal{P(M)}\colon \{M|\, M \subset A\}$, z. B. $\mathcal{P}( \{ 1, 2 \} ) = \{\varnothing, \{1\}, \{2\}, \{1,2\}\}$
\end{center}
Beweis: Wir müssen nur die Existenz einer Bijektion von $\mathcal{E}$ auf $\mathcal{P(\mathbb{N})}$ zeigen. Dazu definiere
\begin{equation*}
\begin{aligned}
	f: \mathcal{E} \to \mathcal{P(\mathbb{N})}\\
	(\epsilon_1, \epsilon_2, \epsilon_3, ...) \mapsto M
\end{aligned}
\end{equation*}
mit $\forall n \in \mathbb{N}: \, (n\in M \Leftrightarrow \mathcal{E}_n = 1)$. Eine solche Abbildung ist offensichtlich bijektiv, wie das Beispiel $f(1,0,1,0,1,0,...) = \{1,3,5,7,...\}$ zeigt. \proofend
\item Sei $W_2\coloneqq \{0,1\}$ die Menge aller endlichen Bitstrings, d.h. aller endlichen 0-1-Folgen. Dann ist $W_2$ abzählbar unendlich.\\
Beweis: Es ist $W_2 = \sqcup_{k=0}^\infty w_2^{(k)}$, wobei $w_2^{(k)}$ die Menge der 0-1-Folgen der Länge k ist. Dann ist $w_2^{(k)}$ endlich mit $|w_2^{(k)}|= 2^k$ und somit:\\
$w_2^{(k)} =\{a_1^{(k)}, a_2^{(k)}, ..., a_{2^k}^{(k)}\}$\\
$W_2 = \{a_1^{(0)}, a_1^{(1)}, a_2^{(1)}, a_1^{(2)}, a_2^{(2)}, a_3^{(2)}, a_4^{(2)}, a_1^{(3)}, a_2^{(3)},...,a_8^{(3)},...\}$
\item $\mathbb{R}$ und $\mathcal{E}$ sind gleichmächtig. (Übungsaufgabe 1.1.c))
\begin{enumerate}
	\item Wir zeigen zunächst, dass $]0,1]$ gleichmächtig zu $\mathcal{E}\setminus\widetilde{W_2}$ mit $\widetilde{W_2}$ ist die Menge aller 0-1-Folgen mit endlich vielen 1en.\\
	Jedes Element von $]0,1]$ besitzt genau eine nicht abbrechende Dualdarstellung $0,b_1, b_2,b_3,...$, mit $b_i \in \{0,1\}$, d. h. mit unendlich vielen 1en.
	\item Wir zeigen, dass $\mathcal{E}$ gleichmächtig zu $\mathcal{E}\setminus\widetilde{W_2}$ ist. Es ist $\widetilde{W_2}$ abzählbar, da sie gleichmächtig zu $W_2$ ist, $\mathcal{E}$ ist nicht abzählbar.\\
	D. h. $\exists \, \widetilde{V_2} \subseteq \mathcal{E}\setminus\widetilde{W_2}$. Seien $\widetilde{W_2} = \{w_1, w_2, w_3, ...\}, \widetilde{V_2} = \{v_1, v_2, v_3,...\}$, dann ist die Abbildung
	\begin{equation*}
		\begin{aligned}
			f\colon \mathcal{E} & \to \mathcal{E}\setminus\widetilde{W_2}\\
			x & \mapsto
		\left\{
		\begin{matrix}
			x, & x \ne \mathcal{E}\setminus\widetilde{W_2}\\
			v_{2i}, & x = v_i\\
			v_{(2i-1)} & x = w_i
		\end{matrix}
		\right .
		\end{aligned}
	\end{equation*}
	\raggedleft
	bijektiv.
	\raggedright
	\item $]0,1]$ und $]0,1[$ sind gleichmächtig, denn sei $U\coloneqq \{1,\frac{1}{2}, \frac{1}{4}, ...\} \subset ]0,1]$, dann ist die Abbildung
	\begin{equation*}
		\begin{aligned}
		f\colon ]0,1] &\to ]0,1[\\
		x & \mapsto
		\left\{
		\begin{matrix}
			x, & falls \, x \ne U\\
			\frac{x}{2}, & falls \, x \in U
		\end{matrix}
		\right .
		\end{aligned}
	\end{equation*}
		\raggedleft
		bijektiv.
		\raggedright		
	\item $\mathbb{R}$ und $]0,1[$ sind gleichmächtig. Betrachte dazu:
	\begin{equation*}
		\begin{aligned}
		f\colon \mathbb{R} &\to \, ]0,1[\\
		x &\mapsto \frac{1}{\pi} (\arctan(x) + \frac{\pi}{2})
		\end{aligned}
	\end{equation*}
	\raggedleft
	eine bijektive Abbildung.
	\raggedright
\end{enumerate}
\item Es existiert eine injektive Abb $f\colon \, ]0,1]\times]0,1] \to ]0,1]$ (man kann sogar zeigen, dass die Mengen gleichmächtig sind) (Übungsaufgabe 1.1.d))\\
Sei $(x,y) \in \, ]0,1] \times ]0,1]$ mit $x = 0,x_1x_2x_3..., y= 0,y_1y_2y_3...$ die eindeutig bestimmten, nicht abbrechenden, Dezimaldarstellungen von x bzw. y. $z\coloneqq 0,x_1y_1x_2y_2...$ sie eine abbrechende Dezimaldarstellung eines Elements $z \in \, ]0,1]$. Die so definierte Abbildung $f\coloneqq \, ]0,1]\times]0,1] \to \, ]0,1], \, (x,y) \mapsto z$ ist offenbar injektiv. Dieses f ist aber nicht surjektiv, da $z = 0,101010... \Rightarrow x=\overline{11}, y = 0$ einen Widerspuch liefert.\\
Man beachte den Satz von Cantor-Bernstein-Schrödinger: Sind $A,B$ Mengen und $f\colon A \to B, \, g\colon B\to A$ injektiv, dann existiert eine bijektive Abbildung $h\colon A \to B$.
\item Für jede Menge A gilt: $\mathcal{P(A)}$ und $A$ sind nicht gleichmächtig. (Es gilt $|\mathcal{P(A)}| > |A|$)
\item Seien $A_1, A_2, A_3, ...$ abzählbare Mengen, dann ist auch deren Vereinigung $\sqcup_{i=1}^\infty A_i$ abzählbar.\\
Denn: Seien O. E. alle $A_i$ unendlich und paarweise disjunkt\\

\parbox{0.4\textwidth}{
	\begin{tabulary}{\textwidth}{c c c}
		$A_1$ & = & $\{a_{11}, a_{12}, a_{13}, ...\}$\\
		$A_2$ & = & $\{a_{21}, a_{22}, a_{23}, ...\}$\\
		$A_3$ & = & $\{a_{31}, a_{32}, a_{33}, ...\}$\\
		$A_4$ & = & $\{a_{41}, a_{42}, a_{43}, ...\}$\\
		$\vdots$ & & $\vdots$
	\end{tabulary}}
\parbox{0.6\textwidth}{
	Nummeriert man diese mit dem Diagonalverfahren, so folgt die Abzählbarkeit.\vspace{45pt}}
\end{itemize}
\subsection*{Topologische Grundbegriffe}
\begin{itemize}
	\item Eine Menge $A \subset \mathbb{R}^n$ heißt \underline{beschränkt}, falls
	\[\exists_{c \in \mathbb{R}^n, r \in \mathbb{R}_{>0}} A \subset B_r(c),\] wobei \[B_r(c)\coloneqq \{x \in \mathbb{R}^n| ||x-c|| < r\}\] die offene n-dimensionale Kugel um $c$ mit Radius $r$ ist.
	\item $A \subset \mathbb{R}^n$ heißt \underline{offen} genau dann, wenn gilt:
	\[\forall_{a \in A} \exists_{\epsilon > 0} B_{\epsilon}(a) \subset A.\]
	\item $A \subset \mathbb{R}^n$ \underline{abgeschlossen} genau dann, wenn gilt:
	\[\mathbb{R} \setminus A \, \textrm{ist offen} \Leftrightarrow \, \forall_{a \in \mathbb{R}^n \setminus A} \exists_{\epsilon > 0} B_{\epsilon}(a) \cap A = \emptyset\]
	\begin{equation*}
		\begin{matrix}
		\textrm{\underline{Allgemein:} für } A \subset \mathbb{R}^n & \textrm{A ist abgeschlossen} & \Leftrightarrow & \mathbb{R}^n \setminus A \, \textrm{ist offen}\\
		& \textrm{A ist offen} & \Leftrightarrow & \mathbb{R}^n \setminus A \, \textrm{ist abgeschlossen}
		\end{matrix}\\
	\end{equation*}
	Einige Beispiele:
	\begin{itemize}
		\item[-] $]0,1] \times \,]0,1] \subset \mathbb{R}^2$ ist weder offen noch abgeschlossen.
		\item[-] Sind $U_i \subset \mathbb{R}^n (i \in I)$ offene Mengen, so ist auch $\bigcup_{i \in I} U_i$ offen im $\mathbb{R}^n$.
		\item[-] Sind $A_i \subset \mathbb{R}^n (i \in I)$ abgeschlossene Mengen, so ist auch $\bigcap_{i \in I} A_i$ abgeschlossen im $\mathbb{R}^n$.
		\item[-] Sind $U_1, U_2 \subset \mathbb{R}^n$ offen, so auch $U_1 \cap U_2$.
		\item[-] Sind $A_1, A_2 \subset \mathbb{R}^n$ abgeschlossen, so auch $A_1 \cup A_2$.
		\item[-] Seien $U_k\coloneqq B_{1+\frac{1}{k}} (0) \subset \mathbb{R}^n, k=1,2,...$ offene Bälle um die Null, dann gilt:
		\[\bigcap_{k=1}^\infty U_k = \{x \in \mathbb{R}\, |\, \Vert x\Vert \leq 1\} = \overline{B_1(0)}\]
	\end{itemize}
	\item Eine Menge $K \subset \mathbb{R}^n$ heißt \underline{kompakt} in topologischen Räumen, falls sie abgeschlossen und beschränkt ist.\\
	Für  ein kompaktes $K \subset \mathbb{R}^n$ gilt: Ist $(U_i)_{i \in I}$ eine Familie offener Mengen, das heißt $U_i \subset \mathbb{R}^n$ ist offen und gilt $K \subset \bigcup_{i \in I} U_i$, dann existiert ein $J \subset I$ mit $|J| \le \infty$ und $K \subset \bigcup_{i \in J} U_i$.\\
	
	In Worten heißt das, dass jede offene Überdeckung einer kompakten Menge eine endliche Teilmenge besitzt.\\
	\underline{Behauptung:}
		Sei $K \subset \mathbb{R}^n$ mit $A_i \subset K \hspace{0.5em} (i \in I)$ abgeschlossene Teilmengen, mit $\bigcap_{i \in I} A_i = \emptyset$, dann folgt:
		\[ \exists_{J \subset I} |J| < \infty \quad \wedge \quad \bigcap_{i \in J} A_i = \emptyset \]
	\underline{Beweis:}
		Sei $U_i \coloneqq \mathbb{R}^n \setminus A_i \quad (i \in I)$, dann folgt
		\begin{align*}
			&\bigcup_{i \in I} U_i = \bigcup_{i \in I} (\mathbb{R}^n \setminus A_i) = \mathbb{R}^n \setminus \bigcap_{i \in I} A_i = \mathbb{R}^n \setminus \emptyset = \mathbb{R}^n \supset K\\
			\Rightarrow\quad &\exists_{J \subset I}\colon |J| < \infty \quad\wedge\quad K \subset \bigcup_{i \in J} U_i \\
			\Rightarrow\quad &K \subset \bigcup_{i \in J} (\mathbb{R}^n \setminus A_i) = \mathbb{R}^n \setminus \underbrace{\bigcap_{i \in J} A_i}_{\subseteq K}\\
	\Rightarrow\quad &\bigcap_{i\in J}A_i=\emptyset
		\end{align*}
 \end{itemize}

\subsection{Lebesgue-Nullmengen}
Ein n-dimensionaler Quader $Q \subset \mathbb{R}^n$ ist ein karthesisches Produkt $Q = ]a_1,b_1[ \times ]a_2,b_2[ \times ... \times ]a_n,b_n[$ offener Intervalle mit $a_i,b_i \in \mathbb{R}$ und $a_i \leq b_i (i=1,...,n)$. Insbesondere ist damit $Q = \emptyset$ ein Quader.
\[ \mu(Q)\coloneqq (b_1 - a_1)(b_2 - a_2)...(b_n - a_n)\, \text{nennen wir das Maß von $Q$}\]

Ist $M = Q_1 \dot\cup Q_2 \dot{\cup} ... \dot{\cup Q_k}$ die disjunkte Vereinigung n-dimensionaler Quader $Q_i$ $(i=1,2,...k)$, dann definieren wir $\mu(M)\coloneqq \sum_{i=1}^{k} \mu(Q_i)$

\underline{Bermerkung:}
Wir haben Quader als offene Menge definiert. Falls wir Randpunkte zulassen, werden wir das als Ausnahme kennzeichnen.
\underline{Sonderfälle:}
\begin{equation*}
	\begin{matrix}
		n=1 &\text{ Quader sind offene Intervalle des } \mathbb{R}, & \mu(Q) \text{ entspricht der Intervalllänge}\\
		n=2 & \text{Quader sind offene Rechtecke, } & \mu(Q) \text{ entspricht dem Flächeninhalt von Q}\\
		n=3 &\text{ Quader, } & \mu(Q) \text{ entspricht dem Volumen von Q}\\
	\end{matrix}
\end{equation*}
\subsubsection{Definition Nullmenge}
Eine Menge $N \subset \mathbb{R}^n$ nennt man n-dimensionale (Lebesgue-)Nullmenge, falls gilt:
\[\forall_{\epsilon > 0} \exists_{Q_i \in \mathbb{R}^n} N \subset \bigcup_{i =1}^{\infty} \text{ und } \sum_{i=1}^{\infty} \mu(Q_i) < \epsilon\]

\underline{Bemerkungen:}
\begin{itemize}
	\item Da $Q_i = \emptyset$ erlaubt ist, sind auch Überdeckungen durch endlich viele Quader erlaubt.
	\item In der Definition könnte man Quader mit Randpunkten zulassen, das wird zur gleichen Klasse von Nullmengen führen.
	\item Warnung: Die Definition hängt von $n$ ab, zum Beispiel ist $N\coloneqq [0,1] \subset \mathbb{R}^1$ keine 1- dimensionale Nullmenge.\\
	Aber: $\tilde{N}\coloneqq[0,1]\times\{0\}\subset\mathbb{R}^2$ ist eine 2-dimensionale Nullmenge, da für $\epsilon > 0 \, \tilde{N}\subset Q_{\epsilon}\coloneqq\, ]-1,2[ \times ]\frac{-\epsilon}{12}, \frac{\epsilon}{12}[ \text{ mit } \mu(Q_{\epsilon}) = \frac{\epsilon}{2} < \epsilon$ gilt.
\end{itemize}
\underline{Beispiele}
\begin{itemize}
	\item Jede abzählbare Menge $M = \{x_1, x_2, x_3, ...\} \subset \mathbb{R}^n$ ist eine n-dimensionale Nullmenge.\\
	Denn: Sei $\epsilon > 0$, dann definiere $Q_i \coloneqq\, ]x_i - \frac{\epsilon}{2^{i+2}}, x_i + \frac{\epsilon}{2^{i+2}}[$ alsdann gilt $\mu(Q_i) = \frac{\epsilon}{2^{i+2}} und M \subset \bigcup_{i=1}^{\infty} Q_i$ und $\sum_{i=1}^{\infty} \mu(Q_i) = \sum_{i=1}^{\infty} \frac{\epsilon}{2^{i+1}} = \frac{\epsilon}{2} < \epsilon$
	\item \underline{Cantorsches Diskontinuum}\\
	Jede Zahl $x \in [0,1]$ lässt sich 3-adisch darstellen als $0,x_1 x_2 x_3 ...$ mit $x_i \in \{0,1,2\}$\\
	\underline{Bemerkung}
	Die Darstellung ist analog zum Dezimalsystem nicht immer eindeutig. Beispielsweise $0,101000... = 0,1002222...$
	\[C\coloneqq \{x \in [0,1] | \exists 3-adische Darstellung x = 0,x_1x_2... \text{ mit } x_i \neq 1 \forall_{i \in \mathbb{N}}\}\] heißt Cantorsches Diskontinuum und ist eine Nullmenge mit übersabzählbar vielen Elementen.\\
	Anschauliche Konstruktion:
\begin{equation*}
	\begin{matrix}
		C_0 = & [0,1]\\
		C_1 = & C_0 \setminus \text{ „offenes Mitteldrittel“ } = [0,\frac{1}{3}] \cup [\frac{2}{3}, 1]\\
		C_2 = & [0,\frac{1}{9}] \cup [\frac{2}{9}, \frac{1}{3}] \cup [\frac{8}{9}, 1]\\
		\vdots & \vdots\\
		C_{n+1} = & C_n \setminus \text{„offenes Mitteldrittel“}
	\end{matrix}
\end{equation*}
Damit folgt:
\begin{itemize}
	\item $C = \bigcap_{i=0}^{\infty} C_i$
	\item $C$ ist abgeschlossen, da alle $C_i$ abgeschlossen sind
	\item $C \subset [0,1] \Rightarrow C $ ist beschränkt und somit insbesondere kompakt.
	\item $C$ ist eine Nullmenge, da $C \subset C_i$ für alle i und $C_i$ ist Vereinigung von Intervallen mit Gesamtlänge $\mu(C_i) = (\frac{2}{3})^{i-1} < \epsilon$ für i groß genug gewählt.
	\item $C$ ist überabzählbar, denn $f\colon C\to \{0,2\}, x\mapsto (x_1 x_2 x_3 ...)$ ist bijektiv und $\{0,2\}^{\mathbb{N}}$ ist überabzählbar, da $\{0,1\}^{\mathbb{N}}$ überabzählbar ist.
\end{itemize}
\end{itemize}

\subsubsection{Proposition}
\begin{itemize}
	\item Jede Teilmenge $N' \subset N$ einer n-dim. Nullmenge $N \subset \mathbb{R}^n$ ist eine n-dim Nullmenge.\item Sind $N_1, N_2, N_3, ... \subset \mathbb{R}^n$ n- dim. Nullmengen, dann ist auch $N = \bigcup_{k=1}^\infty N_k$ n-dim. Nullmengen.
\end{itemize}
\underline{Beweis}
\begin{itemize}
	\item klar nach Definition
	\item Sei $\epsilon > 0$. Nach Voraussetzung existieren Quader $Q_{k,i} \, (k,i) \in \mathbb{N}$ mit:
	
	\begin{equation*}
		\begin{matrix}
		N_1 \subset & \bigcup_{i =1}^{\infty} Q_{1,i}, & \sum_{i=1}^{\infty} \mu(Q_{1,i}) < \frac{\epsilon}{2}\\
		N_2 \subset & \bigcup_{i =1}^{\infty} Q_{2,i}, & \sum_{i=1}^{\infty} \mu(Q_{2,i}) < \frac{\epsilon}{4}\\
		N_3 \subset & \bigcup_{i =1}^{\infty} Q_{3,i}, & \sum_{i=1}^{\infty} \mu(Q_{3,i}) < \frac{\epsilon}{8}\\
		& \vdots
		\end{matrix}\\
	\end{equation*}	
	Also: $N = \bigcup_{k=1}^{\infty} N_k \subset \bigcup_{k,i \in \mathbb{N}} Q_{k,i}, \sum_{k=1}^{\infty}\sum_{i=1}^{\infty} \mu(Q_{k,i}) < \sum_{k=1}^{\infty} \frac{\epsilon}{2^k} = \epsilon$
	\proofend
\end{itemize}

\subsubsection{Wiederholung Riemann-Integral}
Eine Funktion $f \colon [a,b] \to \mathbb{R}$ heißt Riemann-integrierbar, falls es eine Folge von Treppenfunktionen $\varphi_n, \psi_n\colon [a,b] \to \mathbb{R} \, ( n = 1,2,3,...) \text{ mit }$\\ \[
\varphi_n \leq f \leq \psi_n \text{ und } \lim_{n \to \infty} \int_{a}^{b} \varphi_n(x) \,\text{d}x = \lim_{n \to \infty} \int_{a}^{b} \psi_n(x) \, \text{d}x = J\]
gibt.\\
Man definiert dann $\int_{a}^{b} f(x) \text{d}x = J$. Eine Treppenfunktion beschreibt dabei eine Abbildung folgender Art:
\[\varphi\colon [a,b] \to \mathbb{R} \text{, mit } \exists a = x_0<x_1<x_2<...<x_n=b \] derart, dass für alle $k=1,2,...,n \, \varphi$ konstant auf $]x_{k-1}, x_k[$ ist.\\
$\int_{a}^{b}\varphi(x) \text{d}x = \sum_{k=1}^{n} \varphi(\xi_k)(x_k - x_{k-1})$ mit $\xi_k \in ]x_{k-1}, x_1[$ beliebig.\\
Diesen Integralsbegriff werden wir zu den Lebesgue-Integralen erweitern.\\
\underline{Bemerkung:} Es gilt $f\colon [a,b] \to \mathbb{R}$ ist Riemann-integrierbar genau dann, wenn die Menge $U \subset [a,b]$ der Unstetigkeitsmengen von $f$ eine Nullmenge und f beschränkt ist.\\
Dazu zwei Beispiele:\\
\\
Dirichletsche Sprungfunktion:\\
$\begin{matrix}
		f\colon [a,b] & \to & \mathbb{R}\\
		x & \mapsto & \left \{
		\begin{matrix}
			1 & x \text{ rational}\\
			0 & \text{sonst}
		\end{matrix}
		\right .
\end{matrix}$\\
$U = [0,1]$ ist keine Nullmenge, also $f$ nicht Riemann-integrierbar.\\

Die Thomae-Funktion:\\
$\begin{matrix}
	f\colon [0,1] & \to & \mathbb{R}\\
	x & \mapsto & \left \{
	\begin{matrix}
		\frac{1}{q} & x = \frac{p}{q} \, p,q \in \mathbb{N}, ggT(p,q)=1\\
		0 & x \in \mathbb{Q}
	\end{matrix}
	\right .
\end{matrix}$\\
Hier gilt: $U = [0,1] \cap \mathbb{Q}$


\subsection{Das Lebesgue-Integral}

Das Ziel ist eine größere Klasse integrierbarer Funktionen, als beim Riemannintegral. Dazu sei vorab erwähnt, dass es bei jedem plausiblen Integralsbegriff nichtintegrierbare Funktionen $f\colon [a,b] \to \mathbb{R}$ gibt.\\
\underline{Behauptung:} Es gibt keine Funktion (Maß) $\mu\colon P([0,1]) \to [0,1]$ derart, dass
\begin{itemize}
	\item $\mu([0,1]) = 1, \, \mu(\emptyset) = 0$
	\item Ist $A = A_1 \dot{\cup A_2} \dot{\cup} ... \,(A_i \subset [0,1])$, dann $\mu(A) = \sum_{i=1}^{\infty} \mu(A_i)$
	\item Für $A \subset [0,1], \alpha \in [0,1], A_{\alpha} \coloneqq \{y + \alpha | y \in A\} \subset [0,1]$ dann $\mu(A_{\alpha}) = \mu(A)$
\end{itemize}
\underline{Beweis:} Für $x,y\in [0,1]$ definiere $x \sim y \Leftrightarrow x-y \in \mathbb{Q}$\\
Das heißt $[0,1]$ zerfällt in Äquivalenzklassen und es gibt ein Repräsentantensystem $R \subset [0,1]$. $R$ enthält aus jeder Äquivalenzklasse genau ein Element.\\
Für $q\in \mathbb{Q} \cap [0,1[$ definiere $R_q = \{y + q | y \in R, y \in [0,1-q]\} \dot{\cup} \{ y+q-1|y \in ]1-q,1] \}$.\\
Dann gilt: $[0,1] = \dot{\bigcup_{q\in[0,1[}} R_q$\\
Denn die $R_q$ sind paarweise disjunkt, sonst gäbe es zwei Elemente $x\neq y$ von R mit $x-y \in \mathbb{Q}$ und für jedes $x \in [0,1]$ existiert ein $y \in R$ mit $x-y \in \mathbb{Q} \Rightarrow x = y + q \lor x = y + q - 1$ für $q\in [0,1[$ geeignet gewählt.\\
Aber: $1 = \mu([0,1]) = \sum_{q \in [0,1[} \mu(R_q) = \sum_{q \in [0,1[} \mu(R) =
\left \{
\begin{matrix}
0\\
\infty
\end{matrix}
\right . \lightning$\\
\underline{Folgerung:} Ein vernünftiger Integralbegriff für alle Funktionen $f\colon [0,1] \to \mathbb{R}$ sollte insbesondere für Indikatorfunktionen von Mengen $A \subset [0,1]$ einen Wert $\mu(A)\coloneqq \int_{a}^{b} I_A(x) \text{d}x$ liefern, mit $I_A(x) \coloneqq \left \{ \begin{matrix}
1, & x \in A\\
0, & x \ne A
\end{matrix}
\right .$ derart, dass obige Eigenschaften erfüllt sind. Also kann kein solches „universelles“ Integral existieren.

\subsubsection{Definition}
\begin{itemize}
	\item[a)] Sei $D \subseteq \mathbb{R}^n$, dann heißt $\varphi :D \to \mathbb{R}^n$ Treppenfunktion auf D, falls
		\[ \exists \text{Quader} Q_1, Q_2, ..., Q_l \subseteq \mathbb{R}^n \text{paarweise disjunkt mit} \overline{D} \supseteq \bigcup\limit_{i=1}^l \overline{Q_i}
		\text{derart, dass} \varphi \text{für jedes} i = 1,2,...,l \text{konstant auf dem Quader} Q_i \text{ist und} f(x) = 0 \forall x \in D\setminus \bigcup\limit_{i=1}^l\]
	\item[b)] Das Integral der Treppenfuktion \varphi\colon D \to \mathbb{R} aus a) ist definiert durch
		\[ \int_D \varphi(x) \text{d}x = \sum\limit_{i=1}^l \varphi(\xi_i) \mu(Q_i)\]
		wobei $\xi_i \in Q_i (i=1,...,l)$ beliebige Stützstellen sind. 
\end{itemize}

\underline{Bemerkung}
\begin{itemize}
	\item[1)] Für $n=1$ ist das äquivalent zur Definition aus AN 1/2
	\item[2)] Integraldefinition ist wohldefiniert, insbesondere unabhängig von der gewählten Unterteilung im Quader $Q_i$
	\item[3)] Menge der Treppenfuktionen auf D ist abgeschlossen unter Linearkombination
	\[(\varphi, \psi \text{Treppenfuktionen auf D} \alpha, \beta \in \mathbb{R} \Rightarrow \alpha\varphi + \beta\psi \text{Treppenfuktionen auf D})\]
	und unter Max/Min.
	\[(x \mapsto Max (\varphi(x), \psi(x)) \text{ist Treppenfunktion auf D})\]
	\item[4)] Ränder von n-dim Quadern sind n-dim Nullmengen.
\end{itemize}
Im Folgenden ist zunächst meist $D=\overline{Q}$, $Q$ ist Quader, d. h. D ist abgeschlossener Quader oder $D=\mathbb{R}^n$\\
Das ist im Folgenden definierte \underline{Lebesgue-Integral} ist im Vergleich zum Riemannintegral
\begin{itemize}
	\item allgemeiner (mehr integrierbare Funktionen)
	\item hat "schönere" (Konvergenz-) Eigenschaften
\end{itemize}

\underline{Idee:} Ähnlich wie beim Riemannintegral f durch Treppenfuktionen approximieren.
\underline{Sprechweise:} Sei $D \subseteq \mathbb{R}^n$, wir sagen, eine Eigenschaft (z. B. f stetig bei $x \in D$) ist fast überall (f.ü.) erfüllt, falls
\[N=\{x \in D | x \text{erfüllt die Eigenschaft nicht}\}\]
eine (n-dim) Nullmenge ist.

\underline{Schlüsselidee:} punktweise Konvergenz $\varphi_n \longrightarrow^{n \to \infty} f$ f. ü.\\
	(d.h. \exists N \subseteq D \text{Nullmenge:} \forall x \in D \setminus N \colon \lim_{n \to \infty} \varphi_n(x)=f(x)) und damit
	\[\int_D f(x) \text{d}x = \lim_{n \to \infty} \int_D \varphi_n(x) \text{d}x\]
\subsubsection{Definition}
\begin{itemize}
	\item Sei $D\subset \mathbb{R}^n$, dann heißt $\varphi \colon D \to \mathbb{R}$ Trerppenfunktion auf D, falls ein Quader $Q_1, Q_2,..., Q_l \subset \mathbb{R}^n$, paarweise disjunkt mit $\overline{D} \supset \bigcup_{i=1}^l \overline{Q_i}$ derart, dass $\varphi$ für jedes $i = 1,2,...,l$ konstant auf dem Quader $Q_i$ ist und $f(x) = 0$ für alle $x \in D \setminus \bigcup_{i=1}^l Q_i$
	\item Das Integral obiger Treppenfunktion $\varphi \colon D \to \mathbb{R}$ ist definiert durch $\int_{D} \varphi(x) \text{d}x = \sum_{i= 1}^{l} \varphi(\xi_i)\mu(Q_i)$, wobei $\xi \in Q_i (i=1,...,l)$ beliebige Stützstellen sind.

\end{itemize}


\end{document}
