\documentclass[ngerman]{scrartcl}
\usepackage[utf8]{inputenc}
\usepackage[ngerman]{babel}
\usepackage{lmodern} 
\usepackage{graphicx}
\usepackage{amsmath}
\usepackage[amsmath,amsthm,thmmarks]{ntheorem}
\newcommand{\proofend}{\begin{flushright}$\Box$\end{flushright}}
\usepackage{amssymb}
\usepackage{mathtools}
\usepackage{tabulary}
\usepackage{booktabs}
\usepackage{enumitem}
\usepackage{stmaryrd}
\usepackage{wasysym}
\usepackage{hyperref}

\newcommand{\RM}[1]{\MakeUppercase{\romannumeral #1{.}}}
\newcommand{\integ}[2]{\int_{#1} #2 \,\text{d}x}


\theoremstyle{definition}
\newtheorem{definition}[subsubsection]{Definition}
\newtheorem{proposition}[subsubsection]{Proposition}
\newtheorem{satz}[subsubsection]{Satz}
\newtheorem{korollar}[subsubsection]{Korrolar}
\newtheorem*{folgerung}{Folgerung}

\theoremstyle{definition}
\newtheorem{claim}[subsubsection]{Behauptung}
\newtheorem*{claim_n}{Behauptung}

\theoremstyle{remark}
\newtheorem*{remark}{Bemerkung}

\begin{document}
\title{Analysis $\RM{3}$}

\section*{Einführung}
Dieses Skript basiert auf einer Mitschrift der Vorlesung Analysis $\RM{3}$, des Wintersemesters 18/19 an der OTH - Regensburg, gehalten von Herrn Prof. Dr. Illies. Das allgemeine Ziel der Vorlesung ist die Integration im $\mathbb{R}^n$, genauer Gebiets-, Kurven- und Oberflächenintegrale, sowie die dazugehörenden Integrationssätze von Gauß, Stokes und Green.\\
Um dies zu erreichen werden wir das Riemannintegral im $\mathbb{R}^1$ zu dem Lebesgueintegral verallgemeinern. Dabei stellen wir fest, dass das Riemannintegral ein Spezialfall des Lebesgueintegrals ist und mit den Sätzen der majorisierten und der monotonen Konvergenz, dass zweiteres bessere Konvergenzeigenschaften hat.

\newpage

%\section{Lebesgue-Integration oder Gebietsintegrale im $\mathbb{R}^n$}
\subsection{Wiederholung: Abzählbare und endliche Mengen}
\underline{Definition:}
\begin{itemize}[noitemsep]
	\item Zwei Mengen $A$ und $B$ heißen \underline{gleichmächtig}, falls eine bijektive Abbildung \mbox{$f: A\rightarrow B$} existiert.\\
	\item Eine Menge $A$ heißt \underline{abzählbar unendlich}, falls $A$ gleichmächtig zu $\mathbb{N}$ ist.\\
	\item $A$ heißt \underline{abzählbar}, falls $A$ endlich oder abzählbar unendlich ist.\\
	\item $A$ heißt \underline{überabzählbar}, falls $A$ nicht abzählbar ist.
\end{itemize}
\underline{Beispiele:}
\begin{itemize}[noitemsep]
	\item $\mathbb{N}$ ist abzählbar (unendlich)
	\item $\mathbb{Z}$ ist abzählbar (unendlich)
	\item $\mathbb{Q}$ ist abzählbar (unendlich)
	\item Sind $A = \{a_1, a_2, ...\}$ und $B = \{b_1, b_2, ...\}$ abzählbar unendlich, so auch \mbox{$A \cup B = \{a_1, b_1, a_2, b_2, ...\}$}
	\item Falls $A \subset B$ und $A$ überabzählbar, so ist auch $B$ überabzählbar.
	\item $a\le b \Rightarrow [a,b[$ überabzählbar\\
Beweisidee: Cantors Diagonalargument O. E. sei $a = 0, b = 1$, da für $a \leq b$ eine Bijektion $g:[a,b[ \rightarrow [0,1[$ existiert.\\
Angenommen: $[0,1[ = \{a_1, a_2, a_3, ...\}$, O. E. $\alpha_1 = 0, a_1^{(i)} = 0$\\

Seien

\begin{centering}
	\begin{tabulary}{\textwidth}{l l}
		$a_1=0,$ & $\alpha_1^{(1)}, \alpha_2^{(1)}, \alpha_3^{(1)}, ...$\\
		$a_2=0,$ & $\alpha_1^{(2)}, \alpha_2^{(2)}, \alpha_3^{(2)}, ...$\\
		$a_3=0,$ & $\alpha_1^{(3)}, \alpha_2^{(3)}, \alpha_3^{(3)}, ...$\\
		$\vdots$ &
	\end{tabulary}\\
\end{centering}
die nicht abbrechende Dezimaldarstellung der $a_j$, für $i\geq 2$, d. h. $\alpha_i^{(j)} \in \{0,1,...,9\}$ und für $j \geq 2$ sind unendlich viele der $\alpha_1^{(j)}, \alpha_2^{(j)},... \neq 0$

\begin{equation*}
	Sei \, \beta_i = \left\{
		\begin{aligned}
		1, \, & falls \, a_i^{(j)} \neq 1\\
		2, \, & falls \, a_i^{(j)} = 1
		\end{aligned}
		\right.
\end{equation*}
Betrachte die Zahl $\beta \in [0,1[$ mit Dezimaldarstellung $\beta := 0,\beta_1, \beta_2, \beta_3, ...$. Es müsste denmanch ein $i \in \mathbb{N}$ mit $\beta = a_j$ geben. Das hieße, dass für alle $i\in \mathbb{N} \, \beta_i = \alpha_i^{(j)}$ gilt, insbesondere $\beta_i = \alpha_i^{(j)}$ was ein Widerspruch zur Definition unserer $\beta_i$ ist. \proofend

\item $\mathcal{P(\mathbb{N})}$ ist gleichmächtig zu $\{(\epsilon_1, \epsilon_2, \epsilon_3, ...)| \, \epsilon_i \in \{0,1\}\} =: \{0,1\}^\mathbb{N} =: \mathcal{E}$
Es ist demnach $\mathcal{E}$ die Menge aller unendlichen 0-1-Folgen und $\mathcal{P(\mathbb{N})}$ die Potenzmenge von $\mathbb{N}$, wobei für beliebige Mengen $A$ definiert ist:
\begin{center}
	$\mathcal{P(M)}\colon \{M|\, M \subset A\}$, z. B. $\mathcal{P}( \{ 1, 2 \} ) = \{\varnothing, \{1\}, \{2\}, \{1,2\}\}$
\end{center}
Beweis: Wir müssen nur die Existenz einer Bijektion von $\mathcal{E}$ auf $\mathcal{P(\mathbb{N})}$ zeigen. Dazu definiere
\begin{equation*}
\begin{aligned}
	f: \mathcal{E} \to \mathcal{P(\mathbb{N})}\\
	(\epsilon_1, \epsilon_2, \epsilon_3, ...) \mapsto M
\end{aligned}
\end{equation*}
mit $\forall n \in \mathbb{N}: \, (n\in M \Leftrightarrow \mathcal{E}_n = 1)$. Eine solche Abbildung ist offensichtlich bijektiv, wie das Beispiel $f(1,0,1,0,1,0,...) = \{1,3,5,7,...\}$ zeigt. \proofend
\item Sei $W_2\coloneqq \{0,1\}$ die Menge aller endlichen Bitstrings, d.h. aller endlichen 0-1-Folgen. Dann ist $W_2$ abzählbar unendlich.\\
Beweis: Es ist $W_2 = \sqcup_{k=0}^\infty w_2^{(k)}$, wobei $w_2^{(k)}$ die Menge der 0-1-Folgen der Länge k ist. Dann ist $w_2^{(k)}$ endlich mit $|w_2^{(k)}|= 2^k$ und somit:\\
$w_2^{(k)} =\{a_1^{(k)}, a_2^{(k)}, ..., a_{2^k}^{(k)}\}$\\
$W_2 = \{a_1^{(0)}, a_1^{(1)}, a_2^{(1)}, a_1^{(2)}, a_2^{(2)}, a_3^{(2)}, a_4^{(2)}, a_1^{(3)}, a_2^{(3)},...,a_8^{(3)},...\}$
\item $\mathbb{R}$ und $\mathcal{E}$ sind gleichmächtig. (Übungsaufgabe 1.1.c))
\begin{enumerate}
	\item Wir zeigen zunächst, dass $]0,1]$ gleichmächtig zu $\mathcal{E}\setminus\widetilde{W_2}$ mit $\widetilde{W_2}$ ist die Menge aller 0-1-Folgen mit endlich vielen 1en.\\
	Jedes Element von $]0,1]$ besitzt genau eine nicht abbrechende Dualdarstellung $0,b_1, b_2,b_3,...$, mit $b_i \in \{0,1\}$, d. h. mit unendlich vielen 1en.
	\item Wir zeigen, dass $\mathcal{E}$ gleichmächtig zu $\mathcal{E}\setminus\widetilde{W_2}$ ist. Es ist $\widetilde{W_2}$ abzählbar, da sie gleichmächtig zu $W_2$ ist, $\mathcal{E}$ ist nicht abzählbar.\\
	D. h. $\exists \, \widetilde{V_2} \subseteq \mathcal{E}\setminus\widetilde{W_2}$. Seien $\widetilde{W_2} = \{w_1, w_2, w_3, ...\}, \widetilde{V_2} = \{v_1, v_2, v_3,...\}$, dann ist die Abbildung
	\begin{equation*}
		\begin{aligned}
			f\colon \mathcal{E} & \to \mathcal{E}\setminus\widetilde{W_2}\\
			x & \mapsto
		\left\{
		\begin{matrix}
			x, & x \ne \mathcal{E}\setminus\widetilde{W_2}\\
			v_{2i}, & x = v_i\\
			v_{(2i-1)} & x = w_i
		\end{matrix}
		\right .
		\end{aligned}
	\end{equation*}
	\raggedleft
	bijektiv.
	\raggedright
	\item $]0,1]$ und $]0,1[$ sind gleichmächtig, denn sei $U\coloneqq \{1,\frac{1}{2}, \frac{1}{4}, ...\} \subset ]0,1]$, dann ist die Abbildung
	\begin{equation*}
		\begin{aligned}
		f\colon ]0,1] &\to ]0,1[\\
		x & \mapsto
		\left\{
		\begin{matrix}
			x, & falls \, x \ne U\\
			\frac{x}{2}, & falls \, x \in U
		\end{matrix}
		\right .
		\end{aligned}
	\end{equation*}
		\raggedleft
		bijektiv.
		\raggedright		
	\item $\mathbb{R}$ und $]0,1[$ sind gleichmächtig. Betrachte dazu:
	\begin{equation*}
		\begin{aligned}
		f\colon \mathbb{R} &\to \, ]0,1[\\
		x &\mapsto \frac{1}{\pi} (\arctan(x) + \frac{\pi}{2})
		\end{aligned}
	\end{equation*}
	\raggedleft
	eine bijektive Abbildung.
	\raggedright
\end{enumerate}
\item Es existiert eine injektive Abb $f\colon \, ]0,1]\times]0,1] \to ]0,1]$ (man kann sogar zeigen, dass die Mengen gleichmächtig sind) (Übungsaufgabe 1.1.d))\\
Sei $(x,y) \in \, ]0,1] \times ]0,1]$ mit $x = 0,x_1x_2x_3..., y= 0,y_1y_2y_3...$ die eindeutig bestimmten, nicht abbrechenden, Dezimaldarstellungen von x bzw. y. $z\coloneqq 0,x_1y_1x_2y_2...$ sie eine abbrechende Dezimaldarstellung eines Elements $z \in \, ]0,1]$. Die so definierte Abbildung $f\coloneqq \, ]0,1]\times]0,1] \to \, ]0,1], \, (x,y) \mapsto z$ ist offenbar injektiv. Dieses f ist aber nicht surjektiv, da $z = 0,101010... \Rightarrow x=\overline{11}, y = 0$ einen Widerspuch liefert.\\
Man beachte den Satz von Cantor-Bernstein-Schrödinger: Sind $A,B$ Mengen und $f\colon A \to B, \, g\colon B\to A$ injektiv, dann existiert eine bijektive Abbildung $h\colon A \to B$.
\item Für jede Menge A gilt: $\mathcal{P(A)}$ und $A$ sind nicht gleichmächtig. (Es gilt $|\mathcal{P(A)}| > |A|$)
\item Seien $A_1, A_2, A_3, ...$ abzählbare Mengen, dann ist auch deren Vereinigung $\sqcup_{i=1}^\infty A_i$ abzählbar.\\
Denn: Seien O. E. alle $A_i$ unendlich und paarweise disjunkt\\

\parbox{0.4\textwidth}{
	\begin{tabulary}{\textwidth}{c c c}
		$A_1$ & = & $\{a_{11}, a_{12}, a_{13}, ...\}$\\
		$A_2$ & = & $\{a_{21}, a_{22}, a_{23}, ...\}$\\
		$A_3$ & = & $\{a_{31}, a_{32}, a_{33}, ...\}$\\
		$A_4$ & = & $\{a_{41}, a_{42}, a_{43}, ...\}$\\
		$\vdots$ & & $\vdots$
	\end{tabulary}}
\parbox{0.6\textwidth}{
	Nummeriert man diese mit dem Diagonalverfahren, so folgt die Abzählbarkeit.\vspace{45pt}}
\end{itemize}
\subsection{Topologische Grundbegriffe}
\begin{itemize}
	\item Eine Menge $A \subset \mathbb{R}^n$ heißt \underline{beschränkt}, falls
	\[\exists_{c \in \mathbb{R}^n, r \in \mathbb{R}_{>0}} A \subset B_r(c),\] wobei \[B_r(c)\coloneqq \{x \in \mathbb{R}^n| ||x-c|| < r\}\] die offene n-dimensionale Kugel um $c$ mit Radius $r$ ist.
	\item $A \subset \mathbb{R}^n$ heißt \underline{offen} genau dann, wenn gilt:
	\[\forall_{a \in A} \exists_{\epsilon > 0} B_{\epsilon}(a) \subset A.\]
	\item $A \subset \mathbb{R}^n$ \underline{abgeschlossen} genau dann, wenn gilt:
	\[\mathbb{R} \setminus A \, \textrm{ist offen} \Leftrightarrow \, \forall_{a \in \mathbb{R}^n \setminus A} \exists_{\epsilon > 0} B_{\epsilon}(a) \cap A = \emptyset\]
	\begin{equation*}
		\begin{matrix}
		\textrm{\underline{Allgemein:} für } A \subset \mathbb{R}^n & \textrm{A ist abgeschlossen} & \Leftrightarrow & \mathbb{R}^n \setminus A \, \textrm{ist offen}\\
		& \textrm{A ist offen} & \Leftrightarrow & \mathbb{R}^n \setminus A \, \textrm{ist abgeschlossen}
		\end{matrix}\\
	\end{equation*}
	Einige Beispiele:
	\begin{itemize}
		\item[-] $]0,1] \times \,]0,1] \subset \mathbb{R}^2$ ist weder offen noch abgeschlossen.
		\item[-] Sind $U_i \subset \mathbb{R}^n (i \in I)$ offene Mengen, so ist auch $\bigcup_{i \in I} U_i$ offen im $\mathbb{R}^n$.
		\item[-] Sind $A_i \subset \mathbb{R}^n (i \in I)$ abgeschlossene Mengen, so ist auch $\bigcap_{i \in I} A_i$ abgeschlossen im $\mathbb{R}^n$.
		\item[-] Sind $U_1, U_2 \subset \mathbb{R}^n$ offen, so auch $U_1 \cap U_2$.
		\item[-] Sind $A_1, A_2 \subset \mathbb{R}^n$ abgeschlossen, so auch $A_1 \cup A_2$.
		\item[-] Seien $U_k\coloneqq B_{1+\frac{1}{k}} (0) \subset \mathbb{R}^n, k=1,2,...$ offene Bälle um die Null, dann gilt:
		\[\bigcap_{k=1}^\infty U_k = \{x \in \mathbb{R}\, |\, \Vert x\Vert \leq 1\} = \overline{B_1(0)}\]
	\end{itemize}
	\item Eine Menge $K \subset \mathbb{R}^n$ heißt \underline{kompakt} in topologischen Räumen, falls sie abgeschlossen und beschränkt ist.\\
	Für  ein kompaktes $K \subset \mathbb{R}^n$ gilt: Ist $(U_i)_{i \in I}$ eine Familie offener Mengen, das heißt $U_i \subset \mathbb{R}^n$ ist offen und gilt $K \subset \bigcup_{i \in I} U_i$, dann existiert ein $J \subset I$ mit $|J| \le \infty$ und $K \subset \bigcup_{i \in J} U_i$.\\
	
	In Worten heißt das, dass jede offene Überdeckung einer kompakten Menge eine endliche Teilmenge besitzt.
	\underline{Behauptung:}
		Sei $K \subset \mathbb{R}^n$ mit $A_i \subset K (I \in I)$ abgeschlossene Teilmengen, mit $\bigcap_{i \in I} A_i = \emptyset$, dann folgt:
		\[ \exists_{J \subset I} J > \infty \wedge \bigcap_{i \in J} A_i = \emptyset \]
	\underline{Beweis:}
		Sei $U_i \coloneqq \mathbb{R}^n \setminus A_i \, (i \in I)$, dann folgt
		\[ \bigcup_{i \in I} U_i = \bigcup_{i \in I} (\mathbb{R}^n \setminus A_i) = \mathbb{R}^n \setminus \bigcap_{i \in I} A_i = \mathbb{R}^n \setminus \emptyset = \mathbb{R}^n \supset K \]
		\[\Rightarrow \exists_{J \subset I}\colon |J| < \infty \wedge K \subset \bigcup_{i \in J} U_i\]
		\[\Rightarrow K \subset U_{i \in J} (\mathbb{R}^n \setminus A_i) = \mathbb{R}^n \setminus \bigcap_{i \in J} A_i = \emptyset\]
 \end{itemize}

\subsection{Lebesgue-Nullmengen}
Ein n-dimensionaler Quader $Q \subset \mathbb{R}^n$ im $\mathbb{R}^n$ ist ein karthesisches Produkt $Q = ]a_1,b_1[ \times ]a_2,b_2[ \times ... \times ]a_n,b_n[$ offener Intervalle mit $a_i,b_i \in \mathbb{R}$ und $a_i \leq b_i (i=1,...,n)$. Insbesondere ist damit $Q = \emptyset$ ein Quader.
\[ \mu(Q)\coloneqq (b_1 - a_1)(b_2 - a_2)...(b_n - a_n)\] nennen wir das Maß von $Q$

Ist $M = Q_1 \dot\cup Q_2 \dot{\cup} ... \dot{\cup Q_k}$ die disjunkte Vereinigung n-dimensionaler Quader $Q_i$ $(i=1,2,...k)$, dann definieren wir $\mu(M)\coloneqq \sum_{i=1}^{k} \mu(Q_i)$

\underline{Bermerkung:}
Wir haben Quader als offene Menge definiert. Falls wir Randpunkte zulassen, werden wir das als Ausnahme kennzeichnen.
\underline{Sonderfälle:}
\begin{equation*}
	\begin{matrix}
		n=1 &\text{ Quader sind offene Intervalle des } \mathbb{R}, & \mu(Q) \text{ entspricht der Intervalllänge}\\
		n=2 & \text{Quader sind offene Rechtecke, } & \mu(Q) \text{ entspricht dem Flächeninhalt von Q}\\
		n=3 &\text{ Quader, } & \mu(Q) \text{ entspricht dem Volumen von Q}\\
	\end{matrix}
\end{equation*}
\subsubsection{Definition Nullmenge}
Eine Menge $N \subset \mathbb{R}^n$ nennt man n-dimensionale (Lebesgue-)Nullmenge, falls gilt:
\[\forall_{\epsilon > 0} \exists_{Q_i \in \mathbb{R}^n} N \subset \bigcup_{i =1}^{\infty} \text{ und } \sum_{i=1}^{\infty} \mu(Q_i) < \epsilon\]

\underline{Bemerkungen:}
\begin{itemize}
	\item Da $Q_i = \emptyset$ erlaubt ist, sind auch Überdeckungen durch endlich viele Quader erlaubt.
	\item In der Definition könnte man Quader mit Randpunkten zulassen, das wird zur gleichen Klasse von Nullmengen führen.
	\item Warnung: Die Definition hängt von $n$ ab, zum Beispiel ist $N\coloneqq [0,1] \subset \mathbb{R}^1$ keine 1- dimensionale Nullmenge.\\
	Aber: $\tilde{N}\coloneqq[0,1]\times\{0\}\subset\mathbb{R}^2$ ist eine 2-dimensionale Nullmenge, da für $\epsilon > 0 \, \tilde{N}\subset Q_{\epsilon}\coloneqq\, ]-1,2[ \times ]\frac{-\epsilon}{12}, \frac{\epsilon}{12}[ \text{ mit } \mu(Q_{\epsilon}) = \frac{\epsilon}{2} < \epsilon$ gilt.
\end{itemize}
\underline{Beispiele}
\begin{itemize}
	\item Jede abzählbare Menge $M = \{x_1, x_2, x_3, ...\} \subset \mathbb{R}^n$ ist eine n-dimensionale Nullmenge.\\
	Denn: Sei $\epsilon > 0$, dann definiere $Q_i \coloneqq\, ]x_i - \frac{\epsilon}{2^{i+2}}, x_i + \frac{\epsilon}{2^{i+2}}[$ alsdann gilt $\mu(Q_i) = \frac{\epsilon}{2^{i+2}} und M \subset \bigcup_{i=1}^{\infty} Q_i$ und $\sum_{i=1}^{\infty} \mu(Q_i) = \sum_{i=1}^{\infty} \frac{\epsilon}{2^{i+1}} = \frac{\epsilon}{2} < \epsilon$
	\item \underline{Cantorsches Diskontinuum}\\
	Jede Zahl $x \in [0,1]$ lässt sich 3-adisch darstellen als $0,x_1 x_2 x_3 ...$ mit $x_i \in \{0,1,2\}$\\
	\underline{Bemerkung}
	Die Darstellung ist analog zum Dezimalsystem nicht immer eindeutig. Beispielsweise $0,101000... = 0,1002222...$
	\[C\coloneqq \{x \in [0,1] | \exists 3-adische Darstellung x = 0,x_1x_2... \text{ mit } x_i \neq 1 \forall_{i \in \mathbb{N}}\}\] heißt Cantorsches Diskontinuum und ist eine Nullmenge mit übersabzählbar vielen Elementen.\\
	Anschauliche Konstruktion:
\begin{equation*}
	\begin{matrix}
		C_0 = & [0,1]\\
		C_1 = & C_0 \setminus \text{ „offenes Mitteldrittel“ } = [0,\frac{1}{3}] \cup [\frac{2}{3}, 1]\\
		C_2 = & [0,\frac{1}{9}] \cup [\frac{2}{9}, \frac{1}{3}] \cup [\frac{8}{9}, 1]\\
		\vdots & \vdots\\
		C_{n+1} = & C_n \setminus \text{„offenes Mitteldrittel“}
	\end{matrix}
\end{equation*}
Damit folgt:
\begin{itemize}
	\item $C = \bigcap_{i=0}^{\infty} C_i$
	\item $C$ ist abgeschlossen, da alle $C_i$ abgeschlossen sind
	\item $C \subset [0,1] \Rightarrow C $ ist beschränkt und somit insbesondere kompakt.
	\item $C$ ist eine Nullmenge, da $C \subset C_i$ für alle i und $C_i$ ist Vereinigung von Intervallen mit Gesamtlänge $\mu(C_i) = (\frac{2}{3})^{i-1} < \epsilon$ für i groß genug gewählt.
	\item $C$ ist überabzählbar, denn $f\colon C\to \{0,2\}, x\mapsto (x_1 x_2 x_3 ...)$ ist bijektiv und $\{0,2\}^{\mathbb{N}}$ ist überabzählbar, da $\{0,1\}^{\mathbb{N}}$ überabzählbar ist.
\end{itemize}
\end{itemize}

\subsubsection{Proposition}
\begin{itemize}
	\item Jede Teilmenge $N' \subset N$ einer n-dim. Nullmenge $N \subset \mathbb{R}^n$ ist eine n-dim Nullmenge.\item Sind $N_1, N_2, N_3, ... \subset \mathbb{R}^n$ n- dim. Nullmengen, dann ist auch $N = \bigcup_{k=1}^\infty N_k$ n-dim. Nullmengen.
\end{itemize}
\underline{Beweis}
\begin{itemize}
	\item klar nach Definition
	\item Sei $\epsilon > 0$. Nach Voraussetzung existieren Quader $Q_{k,i} \, (k,i) \in \mathbb{N}$ mit:
	
	\begin{equation*}
		\begin{matrix}
		N_1 \subset & \bigcup_{i =1}^{\infty} Q_{1,i}, & \sum_{i=1}^{\infty} \mu(Q_{1,i}) < \frac{\epsilon}{2}\\
		N_2 \subset & \bigcup_{i =1}^{\infty} Q_{2,i}, & \sum_{i=1}^{\infty} \mu(Q_{2,i}) < \frac{\epsilon}{4}\\
		N_3 \subset & \bigcup_{i =1}^{\infty} Q_{3,i}, & \sum_{i=1}^{\infty} \mu(Q_{3,i}) < \frac{\epsilon}{8}\\
		& \vdots
		\end{matrix}\\
	\end{equation*}	
	Also: $N = \bigcup_{k=1}^{\infty} N_k \subset \bigcup_{k,i \in \mathbb{N}} Q_{k,i}, \sum_{k=1}^{\infty}\sum_{i=1}^{\infty} \mu(Q_{k,i}) < \sum_{k=1}^{\infty} \frac{\epsilon}{2^k} = \epsilon$
	\proofend
\end{itemize}

\subsubsection{Wiederholung Riemann-Integral}
Eine Funktion $f \colon [a,b] \to \mathbb{R}$ heißt Riemann-integrierbar, falls es eine Folge von Treppenfunktionen $\varphi_n, \psi_n\colon [a,b] \to \mathbb{R} \, ( n = 1,2,3,...) \text{ mit }$\\ \[
\varphi_n \leq f \leq \psi_n \text{ und } \lim_{n \to \infty} \int_{a}^{b} \varphi_n(x) \,\text{d}x = \lim_{n \to \infty} \int_{a}^{b} \psi_n(x) \, \text{d}x = J\]
gibt.\\
Man definiert dann $\int_{a}^{b} f(x) \text{d}x = J$. Eine Treppenfunktion beschreibt dabei eine Abbildung folgender Art:
\[\varphi\colon [a,b] \to \mathbb{R} \text{, mit } \exists a = x_0<x_1<x_2<...<x_n=b \] derart, dass für alle $k=1,2,...,n \, \varphi$ konstant auf $]x_{k-1}, x_k[$ ist.\\
$\int_{a}^{b}\varphi(x) \text{d}x = \sum_{k=1}^{n} \varphi(\xi_k)(x_k - x_{k-1})$ mit $\xi_k \in ]x_{k-1}, x_1[$ beliebig.\\
Diesen Integralsbegriff werden wir zu den Lebesgue-Integralen erweitern.\\
\underline{Bemerkung:} Es gilt $f\colon [a,b] \to \mathbb{R}$ ist Riemann-integrierbar genau dann, wenn die Menge $U \subset [a,b]$ der Unstetigkeitsmengen von $f$ eine Nullmenge und f beschränkt ist.\\
Dazu zwei Beispiele:\\
\\
Dirichletsche Sprungfunktion:\\
$\begin{matrix}
		f\colon [a,b] & \to & \mathbb{R}\\
		x & \mapsto & \left \{
		\begin{matrix}
			1 & x \text{ rational}\\
			0 & \text{sonst}
		\end{matrix}
		\right .
\end{matrix}$\\
$U = [0,1]$ ist keine Nullmenge, also $f$ nicht Riemann-integrierbar.\\

Die Thomae-Funktion:\\
$\begin{matrix}
	f\colon [0,1] & \to & \mathbb{R}\\
	x & \mapsto & \left \{
	\begin{matrix}
		\frac{1}{q} & x = \frac{p}{q} \, p,q \in \mathbb{N}, ggT(p,q)=1\\
		0 & x \in \mathbb{Q}
	\end{matrix}
	\right .
\end{matrix}$\\
Hier gilt: $U = [0,1] \cap \mathbb{Q}$


\subsection{Das Lebesgue-Integral}

Das Ziel ist eine größere Klasse integrierbarer Funktionen, als beim Riemannintegral. Dazu sei vorab erwähnt, dass es bei jedem plausiblen Integralsbegriff nichtintegrierbare Funktionen $f\colon [a,b] \to \mathbb{R}$ gibt.\\
\begin{claim_n} Es gibt keine Funktion (Maß) $\mu\colon P([0,1]) \to [0,1]$ derart, dass
	\begin{itemize}
		\item $\mu([0,1]) = 1, \, \mu(\emptyset) = 0$
		\item Ist $A = A_1 \dot{\cup} A_2 \dot{\cup} ... \,(A_i \subset [0,1])$, dann $\mu(A) = \sum_{i=1}^{\infty} \mu(A_i)$
		\item Für $A \subset [0,1], \alpha \in [0,1], A_{\alpha} \coloneqq \{y + \alpha | y \in A\} \subset [0,1]$ dann $\mu(A_{\alpha}) = \mu(A)$
	\end{itemize}
\end{claim_n}
\begin{proof} Für $x,y\in [0,1]$ definiere $x \sim y \Leftrightarrow x-y \in \mathbb{Q}$\\
	Das heißt $[0,1]$ zerfällt in Äquivalenzklassen und es gibt ein Repräsentantensystem $R \subset [0,1]$. $R$ enthält aus jeder Äquivalenzklasse genau ein Element.\\
	Für $q\in \mathbb{Q} \cap [0,1[$ definiere 
	\[
		R_q = \{y + q | y \in R, y \in [0,1-q]\} \dot{\cup} \{ y+q-1|y \in ]1-q,1] \}
	\]
	Dann gilt: $[0,1] = \dot{\bigcup}_{q\in[0,1[} R_q$, denn die $R_q$ sind paarweise disjunkt. Sonst gäbe es zwei Elemente $x\neq y$ von R mit $x-y \in \mathbb{Q}$ und für jedes $x \in [0,1]$ existiert ein $y \in R$ mit $x-y \in \mathbb{Q} \Rightarrow x = y + q \lor x = y + q - 1$ für $q\in [0,1[$ geeignet gewählt.\\
	Aber: \[1 = \mu([0,1]) = \sum_{q \in [0,1[} \mu(R_q) = \sum_{q \in [0,1[} \mu(R) =
		\left \{
		\begin{matrix}
			0 \\
			\infty
		\end{matrix}
		\right . \lightning\]
\end{proof}
\underline{Folgerung:} Ein vernünftiger Integralbegriff für alle Funktionen $f\colon [0,1] \to \mathbb{R}$ sollte insbesondere für Indikatorfunktionen von Mengen $A \subset [0,1]$ einen Wert $\mu(A)\coloneqq \int_{a}^{b} I_A(x) \text{d}x$ liefern, mit $I_A(x) \coloneqq \left \{ \begin{matrix}
		1, & x \in A \\
		0, & x \ne A
	\end{matrix}
	\right .$ derart, dass obige Eigenschaften erfüllt sind. Also kann kein solches „universelles“ Integral existieren.

\begin{definition}[Treppenfunktion]\leavevmode
	\begin{itemize}
		\item[a)] Sei $D \subseteq \mathbb{R}^n$, dann heißt $\varphi :D \to \mathbb{R}^n$ Treppenfunktion auf D, falls
		      \[ \exists \text{Quader } Q_1, Q_2, ..., Q_l \subseteq \mathbb{R}^n \text{ paarweise disjunkt mit } \overline{D} \supseteq \bigcup_{i=1}^l \overline{Q_i}\]
		      derart, dass $\varphi$ für jedes $i = 1,2,...,l$ konstant auf dem Quader $Q_i$ ist und \[f(x) = 0 \forall x \in D\setminus \bigcup\lim_{i=1}^l\]
		\item[b)] Das Integral der Treppenfuktion $\varphi\colon D \to \mathbb{R}$ aus a) ist definiert durch
		      \[ \int_D \varphi(x) \text{d}x = \sum_{i=1}^l \varphi(\xi_i) \mu(Q_i)\]
		      wobei $\xi_i \in Q_i (i=1,...,l)$ beliebige Stützstellen sind. 
	\end{itemize}
\end{definition}

\underline{Bemerkung}
\begin{itemize}
	\item[1)] Für $n=1$ ist das äquivalent zur Definition aus AN 1/2
	\item[2)] Integraldefinition ist wohldefiniert, insbesondere unabhängig von der gewählten Unterteilung im Quader $Q_i$
	\item[3)] Menge der Treppenfuktionen auf D ist abgeschlossen unter Linearkombination
	      \[(\varphi, \psi \text{ Treppenfunktionen auf D } \alpha, \beta \in \mathbb{R} \Rightarrow \alpha\varphi + \beta\psi \text{ Treppenfuktionen auf D})\]
	      und unter Max/Min.
	      \[(x \mapsto \text{Max} (\varphi(x), \psi(x)) \text{ ist Treppenfunktion auf D})\]
	\item[4)] Ränder von n-dim Quadern sind n-dim Nullmengen.
\end{itemize}
Im Folgenden ist zunächst meist $D=\overline{Q}$, $Q$ ist Quader, d. h. D ist abgeschlossener Quader oder $D=\mathbb{R}^n$\\
Das ist im Folgenden definierte \underline{Lebesgue-Integral} ist im Vergleich zum Riemannintegral
\begin{itemize}
	\item allgemeiner (mehr integrierbare Funktionen)
	\item hat „schönere“ (Konvergenz-) Eigenschaften
\end{itemize}

\underline{Idee:} Ähnlich wie beim Riemannintegral f durch Treppenfuktionen approximieren.\\

\underline{Sprechweise:} Sei $D \subseteq \mathbb{R}^n$, wir sagen, eine Eigenschaft (z. B. f stetig bei $x \in D$) ist fast überall (f.ü.) erfüllt, falls
\[N=\{x \in D | x \text{ erfüllt die Eigenschaft nicht}\}\]
eine (n-dim) Nullmenge ist.\\

\underline{Schlüsselidee:} punktweise Konvergenz $\varphi_n \xrightarrow[n \to \infty]{}f$ f. ü.\\
(d.h. $\exists N \subseteq D \text{ Nullmenge: } \forall x \in D \setminus N \colon \lim_{n \to \infty} \varphi_n(x)=f(x)$) und damit
\[\int_D f(x) \text{d}x = \lim_{n \to \infty} \left(\int_D \varphi_n(x) \text{d}x\right)\]

Die präzise Definition erfordert mehrere Schritte:

\begin{definition}\leavevmode
	\begin{itemize}
		\item[a)] Sei $D \subseteq \mathbb{R}^n$ ein abgeschlossener Quader oder $D = \mathbb{R}^n$. Dann ist $L^\uparrow(D)$ definiert als Menge aller Funktion $f \colon D \to \mathbb{R}$ derart, dass eine Folge von Treppenfunktionen $\varphi_n \colon D \to \mathbb{R} (m=1,2,...,n)$ die f.ü. monoton wachsend gegen $f$ konvergiert (d.h. die Menge $\{x \in D | (\varphi_n(x))_n\ \text{nicht monoton wachsend}\}$ und $\{x \in D | (\varphi(x)_n)\ \text{konvergiert nicht gegen f(x)}\}$ sind Nullmengen) und für die\\
		      $\lim_{k \mapsto \infty} (\int_D \varphi_k(x) \text{d}x) \in \mathbb{R}$ existiert.
		\item[b)]  In der Situation von a)
		      \[\int_S f(x) \text{d}x \coloneqq \lim_{k \to \infty} \left(\int_D \varphi_k(x) \text{d}x \right)\]
		      also für alle $f \in L^\uparrow(D)$.
	\end{itemize}
\end{definition}

\begin{proposition}
	In der Situation von obiger Definition ist $\int_D f(x) \text{d}x$ für $f\in L^\uparrow(D)$ wohldefiniert. Genauer:\\
	Ist $\tilde{\varphi}_n \colon D \to \mathbb{R} (n=1,2,...)$ eine andere f. ü. monoton wachsend gegen $f$ konvergierende Folge von Treppenfuktionen, dann gilt
	\[\lim_{n \to \infty} \left(\int_D \tilde{\varphi}_n (x) \text{d}x\right) = \lim_{n \to \infty} \left(\int_D \varphi_n(x) \text{d}x\right)\]
\end{proposition}
\underline{Beweis:}
Verwendet:
\subsubsection{Lemma} Sei $D \subseteq \mathbb{R}^n$ abgeschlossener Quader oder $D= \mathbb{R}^n$. Sei $\varphi_n \colon D \to \mathbb{R} (n=1,2,3,...)$ Folge von Treppenfuktionen mit $\varphi_n \geq 0$ f. ü. und die f. ü. monoton fällt und f. ü. gegen 0 konvergiert.\\
Dann gilt:
\[\lim_{k \to \infty} \left(\int_D\varphi_k(x)\text{d}x\right) = 0\]
Den Beweis hierfür kann für n=1 auf Seite 607 im Arens anchgelesen werden.\\

\underline{Vorab:} Die Folgen $\left(\int_D \varphi_k (x) \text{d}x\right)_k, \left(\int_D \tilde{\varphi}_k (x) \text{d}x\right)_k$ sind nach Definition monoton wachsend.\\
Das heißt es genügt zu zeigen: $\forall_{\epsilon > 0} \forall_{k_0 \in \mathbb{N}} \exists_{k_1,k_2 \in \mathbb{N}_{\geq k_0}}$ mit

\begin{align}
	\int_D \varphi_{k_1}(x) \text{d}x > \int_D \tilde{\varphi}_{k_0}(x) \text{d}x - \epsilon \label{eq:Gl1} \\
	\int_D \varphi_{k_2}(x) \text{d}x > \int_D \tilde{\varphi}_{k_0}(x) \text{d}x - \epsilon \label{eq:Gl2}
\end{align}

\underline{Dazu:} Für $k \geq k_0$ betrachte $\varphi_k \colon D \to \mathbb{R}$ definiert durch
\[\psi_k(x) \coloneqq \text{max} \left( \tilde{\varphi}_{k_0}(x) - \varphi_k(x), 0\right) \geq 0\]
Da $\varphi_k \xrightarrow[k \to \infty]{}f , \, \tilde{\varphi_k} \xrightarrow[k \to \infty]{}f$ f. ü. monton, erfüllt $(\psi_n)_n$ die Voraussetzung von obigem Lemma. Also existiert $k_1 \geq k$ mit $\int_D \varphi_{k_1}(x) \text{d}x < \epsilon$\\
\[\Rightarrow \int_D \varphi_{k_1}(x) \text{d}x \geq \int_D \left(\tilde{\varphi}_{k_0} - \varphi_{k_1}\right)(x)\text{d}x\]
\[= \int_D \tilde{\varphi}_{k_0}(x)\text{d}x - \int_D \varphi_{k_1}(x)\text{d}x\]
Damit folgt die Gleichung \ref{eq:Gl1}, \ref{eq:Gl2} wird analog gezeigt.

\subsubsection{Proposition}
Sei $D \subseteq \mathbb{R}^n$ abgeschlossener Quader oder $D = \mathbb{R}^n$ und seien $f,g \in L^\uparrow(D)$
\begin{itemize}
	\item[a)] Sind $\alpha, \beta \in \mathbb{R}_{\geq 0}$, dann gilt auch $\alpha f + \beta g \in L^\uparrow(D)$ und \[\int_D \left(\alpha f(x) + \beta g(x)\right) \text{d}x = \alpha \int_D f(x) \text{d}x + \beta \int_D g(x) \text{d}x\]
	\item[b)] $f \geq g$ f. ü. $\Rightarrow$ \[\int_D f(x) \text{d}x \geq \int_D g(x) \text{d}x\]
	\item[c)] $\text{min}(f,g) \in L^\uparrow (D)$ mit \[\int_D \text{min}(f,g)(x) \text{d}x \geq \text{min} \left(\int_D f(x)\text{d}x, \int_D g(x)\text{d}x\right)\]
\end{itemize}

\underline{Beweisskizze} Es gelte $\varphi_n \xrightarrow[n \to \infty]{} f, \psi_n \xrightarrow[n \to \infty]{}g$ für monoton wachsend mit Treppenfuktionen $\varphi_n, \psi_n \colon D \to \mathbb{R}$
\begin{itemize}
	\item[a)] $\Phi_n \coloneqq \alpha \varphi_n + \beta \psi_n$, dann ist $\Phi_n \colon D \to \mathbb{R}$ Treppenfunktionen und da $\alpha, \beta >0$ gilt: \[\alpha\varphi_n \xrightarrow[n \to \infty]{}\alpha f, \beta\psi_n \xrightarrow[n \to \infty]{}\beta g\] f. ü. monoton wachsend, also \[\alpha \varphi_n + \beta \psi_n \xrightarrow[n \to \infty]{}\alpha f + \beta g\] f. ü. monoton wachsend und \[\integ{D}{\left(\alpha\varphi_n(x) + \beta\psi_n(x)\right)} = \alpha \integ{D}{\varphi_n(x) } + \beta\integ{D}{\psi_n(x)}\] woraus die Behauptung folgt.
	\item[b)] Gelte $f \geq g$ f. ü., sei $\tilde{\varphi}_n \coloneqq \text{max}(\varphi_n, \psi_n)$, dann ist $\varphi_n \colon D \to  \mathbb{R}$ Treppenfunktion mit $\varphi_n \leq \tilde{\varphi}_n \leq f$ f. ü. und $(\tilde{\varphi}_n(x))_n$ f. ü. monoton wachsend, d.h. \[\tilde{\varphi}_n \xrightarrow[n \to \infty]{}f \text{f. ü. monoton wachsend,}\]
	      \[d.h. \integ{D}{f(x)} = \lim_{n \to \infty} \integ{D}{\tilde{\varphi}_n(x)} \geq \lim_{n \to \infty} \integ{D}{\psi_n(x)} = \integ{D}{g(x)}\]
	      nach der Definition.
	\item[c)] $\tilde{\varphi}_n = \text{max}(\varphi_n,\psi_n)$ ist eine Treppenfunktion, $\tilde{\varphi}_n \colon D \to \mathbb{R}$ mit $\tilde{\varphi_n} \xrightarrow[n \to \infty]{} \text{min}(f,g)$ f. ü. monoton wachsend mit \[\integ{D}{\tilde{\varphi}_n(x)} \leq \integ{D}{\varphi_n(x)} \leq \integ{D}{f(x)} \in \mathbb{R}\] Da monoton wachsende beschränkte Folgen reeller Zahlen konvergieren, folgt \[\lim_{n \to \infty} \integ{D}{\tilde{\varphi}_n(x)} \in \mathbb{R} \text{ existiert, d. h. min}(f,g) \in L^\uparrow(D) \text{ mit}\] \[\integ{D}{\text{min}(f,g)(x)} = \lim_{n \to \infty} \integ{D}{\tilde{\varphi}_n(x)} \leq \text{min}\left(\lim_{n \to \infty} \integ{D}{\varphi_n(x)}, \lim_{n \to \infty} \integ{D}{\psi_n(x)}\right)\]
\end{itemize}

\begin{satz}\leavevmode
	\begin{itemize}
		\item[a)] Im Fall $n=1$, $D=[a, b]\ \ (a<b,\, a,b\in\mathbb{R})$ gilt:\\
		Jede Riemann-Int'bare Funktion $f\colon D\to\mathbb{R}$ liegt in $L^\uparrow(D)$ mit \[\integ{D}{f(x)}=\int_a^bf(x) \text{d}x\]
		\item[b)] Sei $D\subseteq \mathbb{R}^n$ ein abgeschlossener Quader, $f\colon D\to\mathbb{R}$ eine beschränkte Funktion die f.ü. stetig ist, so gilt $f\in L^\uparrow(D)$.
	\end{itemize}
\end{satz}
\begin{proof}
	Zuhause
\end{proof}

\underline{Beispiele:}\\
\begin{enumerate}
	\item \begin{itemize}
		\item Dirichlet'sche Sprungfunktion:\\
	\begin{align*}
		f\colon [0,1] &\to \mathbb{R}\\
		x&\mapsto \left\{\begin{matrix}
			1 & \text{falls } x \in\mathbb{Q}\\
			0 & \text{sonst}
		\end{matrix}\right.
	\end{align*}
	Dann gilt $f\in L^\uparrow(D)$ obwohl $f$ überall unstetig ist mit $\integ{D}{f(x)}=0$\\
	Denn mit $\varphi_n\colon [0, 1]\to \mathbb{R};\;x\mapsto 0$ gilt: $\varphi_n$ ist Treppenfunktion auf $D=[0,1]$ und $\varphi_n\xrightarrow[n\to\infty]{}f$ f.ü. monoton wachsend, da $[0,1]\cap\mathbb{Q}$ eine Nullmenge ist.

	\item $g=1-f, g(x)=\left\{\begin{matrix}
		0 & \text{falls }  x \in\mathbb{Q}\\
		1 & \text{sonst}
	\end{matrix}\right.\Rightarrow g\in L^\uparrow(D)\ \text{mit}\ \integ{D}{g(x)}=1$ .
	\end{itemize}
	\item Übung 1.8\\
	???
\end{enumerate}

\begin{definition}[Lebesgue-integrierbar]
	Sei $D\subseteq\mathbb{R}^n$ ein abgeschlossener Quader oder $\mathbb{R}^n$. $f\colon D\to\mathbb{R}$ heißt \underline{Lebesgue-integrierbar} ($f\in L(D)$), falls $\exists g,h\in L^\uparrow(D)$ mit $f=g-h$.\\
	Das Lebesgue-Integral von $f$ ist definiert durch
	\[
		\integ{D}{f(x)}\coloneqq\integ{D}{g(x)}-\integ{D}{h(x)}	
	\]
\end{definition}

\begin{proposition}
	Das Integral $\integ{D}{f(x)}$ ist für $f\in L(D)$ wohldefiniert.
\end{proposition}
\begin{proof}
	Seien $g,\tilde{g},h,\tilde{h}\in L^\uparrow(D)$ mit $f=g-h=\tilde{g}-\tilde{h}$.\\
	Zu zeigen ist \[\integ{D}{g(x)}-\integ{D}{h(x)} = \integ{D}{\tilde{g}(x)}-\integ{D}{\tilde{h}(x)}\]
	Es gilt $g+\tilde{h}=\tilde{g}+h\in L^\uparrow(D)$ mit 
	\begin{align*}
		\integ{D}{g(x)}+\integ{D}{\tilde{h}(x)}
		&=\integ{D}{g(x)+\tilde{h}(x)}\\
		=\integ{D}{\tilde{g}(x)+h(x)}
		&=\integ{D}{\tilde{g}(x)}+\integ{D}{h(x)}\\
		&\Rightarrow \text{Behauptung}
	\end{align*}
\end{proof}
Da für $f\in L^\uparrow(D)$ gilt $f=f-0$ und $-f = 0 - f$, und $0\in L^\uparrow(D)$, folgt $f,-f\in L(D)$, insbesondere $L^\uparrow(D)\subset L(D)$.\\
Weiter $\forall f_1,f_2\in L(D), \alpha, \beta\in\mathbb{R}$ gilt $\alpha f_1 + \beta f_2\in L(D)$ (denn mit $f_1=g_1-h_1;\; f_2=g_2-h_2$ gilt:\\
Hat man $\alpha f_1+\beta f_2=|\alpha|(\underbrace{\pm f_1}_{\in L(D)})+|\beta|(\underbrace{\pm f_2}_{\in L(D)})$. Es genügt also, die Aussage für $\alpha, \beta\geq 0$ zu zeigen.\\
Nun folgt direkt 
\[
	\alpha f_1+\beta f_2 = (\underbrace{\alpha g_1 + \beta g_2}_{\in L^\uparrow(D)}) - (\underbrace{\alpha h_1 + \beta h_2}_{\in L^\uparrow(D)})\in L(D)
\]
$\integ{D}{\alpha f_1(x)+\beta f_2(x)}=\alpha\integ{D}{f_1(x)}+\beta\integ{D}{f_2(x)}$ folgt analog.


\begin{proposition}\leavevmode
	\begin{itemize}
		\item[a)] $f,g\in L(D),\,\alpha,\beta\in\mathbb{R}\Rightarrow (\alpha f+\beta g)\in L(D)$
		und \\$\integ{D}{\alpha f(x)+\beta g(x)} =\alpha\integ{D}{f(x)}+\beta\integ{D}{g(x)}$
		\item[b)] Ist $f\in L(D)$ und $\tilde{f}\colon D\to\mathbb{R}$ mit $f=\tilde{f}$ f.ü., dann gilt $\tilde{f}\in L(D)$ mit $\integ{D}{f(x)}=\integ{D}{\tilde{f}(x)}$
		\item[c)] Sind $f, \tilde{f}\in L(D)$ und $f\leq \tilde{f}$ f.ü., dann gilt $\integ{D}{f(x)} \leq\integ{D}{\tilde{f}(x)}$.
		\item[d)] Mit $f\in L(D)$ ist auch $|f|\in L(D)$ mit $\left|\integ{D}{f(x)}\right|\leq\integ{D}{|f(x)|}$. 
	\end{itemize}
\end{proposition}
\begin{proof}\leavevmode
	\begin{itemize}
		\item[a)] Schon erfolgt.
		\item[b)] Sei $g\coloneqq f-\tilde{f}$, d.h. $g=0$ f.ü. $\Rightarrow g\in L^\uparrow(D)$ mit $\integ{D}{g}=0$ (Wähle $\varphi_n = 0$)\\
		Mit $\tilde{f}=f+g\underset{\text{a)}}{\Rightarrow}\tilde{f}\in L(D)$ mit $\integ{D}{\tilde{f}(x)}= \integ{D}{f(x)}+\underbrace{\integ{D}{g(x)}}_{=0}$.
		\item[c)] Ü 1.9\\
		Seien $f,\tilde{f}\in L(D)$ mit $f\leq\tilde{f}$ f.ü.\\
		$\Rightarrow \exists g,h,\tilde{g},\tilde{h}\in L^\uparrow(D)$ mit $g-h=f,\, \tilde{g}-\tilde{h}=\tilde{f}$.\\
		$\Rightarrow g-h\leq\tilde{g}-\tilde{h}$ f.ü. $\Rightarrow g+\tilde{h} \leq \tilde{g}+h$ f.ü. und $g+\tilde{h},\tilde{g}+h\in L^\uparrow(D)$\\
		\begin{align*}
			\Rightarrow \integ{D}{(g(x)+\tilde{h}(x))}&\leq\integ{D}{(\tilde{g}(x)+h(x))}\\
			\Rightarrow \integ{D}{g(x)}+\integ{D}{\tilde{h}(x)}&\leq\integ{D}{\tilde{g}(x)}+\integ{D}{h(x)}\\
			\Rightarrow \integ{D}{g(x)}-\integ{D}{h(x)}&\leq\integ{D}{\tilde{g}(x)}-\integ{D}{\tilde{h}(x)}\\
			\Rightarrow \integ{D}{(g(x)-h(x))}&\leq\integ{D}{(\tilde{g}(x)-\tilde{h}(x))}\\
		\end{align*}
		\item[d)] Sei $f\in L(D)\Rightarrow\exists g,h\in L^\uparrow(D)\ \text{mit}\ f=g-h$\\
		Man hat $|f| = |g-h| = \underbrace{g+h}_{\in L^\uparrow(D)}-2\underbrace{\min\{g,h\}}_{\in L^\uparrow(D)}\in L(D)$.\\
		Es folgt
		\[\left.\begin{matrix}
			f\leq|f|\Rightarrow \integ{D}{f(x)}\leq\integ{D}{|f(x)|}\\
			-f\leq|f|\Rightarrow -\integ{D}{f(x)}\leq\integ{D}{|f(x)|}
		\end{matrix}\right\}\Rightarrow\left|\integ{D}{f(x)}\right|\leq\integ{D}{|f(x)|}\]
		 
	\end{itemize}
\end{proof}
\subsection{Konvergenzsätze}
Sei $F:\mathbb{R}\to\mathbb{R}\ x\mapsto 0$ und für $n\in\mathbb{N}\; f_n\colon\mathbb{R}\to\mathbb{R}\colon\ x\mapsto\left\{\begin{matrix}
	1 & \text{falls }n\leq x\leq n+1\\
	0 & \text{sonst}
\end{matrix}\right.$\\
$\Rightarrow$ für alle $x\in\mathbb{R}$ gilt $\lim_{n\to\infty}f_n(x)=0=f(x)$. D.h. $f_n\xrightarrow[n\to\infty]{}f$ punktweise.\\
Aber $\integ{\mathbb{R}}{f(x)} = 0, \integ{\mathbb{R}}{f_n(x)} = 1$, also $\integ{\mathbb{R}}{f_n(x)}\nrightarrow\integ{\mathbb{R}}{f(x)}$\\
Bedingungen, unter denen aus $f_n\xrightarrow[n\to\infty]{}f$ folgt $\integ{D}{f_(x)}\xrightarrow[n\to\infty]{}\integ{D}{f(x)}$:\\
\begin{itemize}
    \item Ana 1/2: Für Riemann Integrale braucht man gleichmäßige Konvergenz
    \item Ana 3: Bei Lebesgue Integralen reichen einfachere Bedingungen! (Satz von monotonen bzw. majorisierten Konvergenz)
\end{itemize}
\begin{satz}[Monotone Konvergenz, B. Levi]
    $D\subset\mathbb{R}^n$ abg. Quader oder $D=\mathbb{R}^n$. Ist $f_n\colon D\to\mathbb{R}$ Folge von Funktionen, die f.ü. monoton wachsen. Die Folge der Integrale $\integ{D}{f_n(x)}$ sei beschränkt. Dann konvergiert $(f_n)_n$ f.ü. punktweise gegen ein $f\in L(D)$ mit $\lim_{n\to\infty}\integ{D}{f_n(x)}=\integ{D}{f(x)}$.
\end{satz}
\begin{proof}
    Vorab nach Proposition \ref{integ_rechenregeln} ist $\integ{D}{f_n(x)}$ monoton wachsend, aus der Beschränktheit folgt somit die Konvergenz 
    \[
        \lim_{n\to\infty}\integ{D}{f_n(x)}\in\mathbb{R}
    \]
    (Arens S. 625)\\
    Für $f_n\in L^\uparrow(D)$ ist der Beweis "leicht". Für $f_n\in L(D)$, d.h. $f_n=g_n-h_n$ mit $g_n,h_n\in L^\uparrow(D)$ kann man zeigen, dass $h_n$ "klein gewählt" werden kann; damit Rückführung auf $L^\uparrow(D)$-Fall.
\end{proof}
Beispiel:\\
Sei $f\colon [0,1]\to\mathbb{R};\; x\mapsto\left\{\begin{matrix}
    \sin\left(\frac{1}{x}\right)& x\neq 0\\
    0 & x = 0
\end{matrix}\right.$\\
Frage: $\exists$ Lebesgue-Integral $\integ{[0,1]}{f(x)}$?\\
Betrachte für $n\in\mathbb{N}\; f_n:[0,1]\to\mathbb{R};\;x\mapsto\left\{\begin{matrix}
    f(x) & x\geq \frac{1}{n}\\
    -1 & sonst
\end{matrix}\right.\\\Rightarrow \forall x\in (0,1]:\;f_n(x)\xrightarrow[n\to\infty]{}f(x)$ und $(f_n(x))_n$ monton wachsend für alle $x\in [0,1]$.\\
Bemerkung:\\
Tatsächlich sieht man sofort:
\[\integ{[0,1]}{f(x)} = \lim_{n\to\infty}\int_{\frac{1}{n}}^1f(x)\,\text{d}x + \lim_{n\to\infty}\int_0^{\frac{1}{n}}(-1)\,\text{d}x = \lim_{n\to\infty}\int_{\frac{1}{n}}^1f(x)\,\text{d}x\]
ist uneigentliches Riemann-Integral $\int_0^1f(x)\,\text{d}x$, dessen Existenz hiermit gezeigt ist.

\begin{korollar}
    Sei $D$ wie zuvor. Sei $f\colon D\to\mathbb{R}$ und $D_1\subseteq D_2\subseteq\cdots\subseteq D$ aufsteigende Folge abg. Quader mit $\bigcup_{m=1}^\infty D_m = D$. Dann gilt:
    \begin{align*}
        f\in L(D)\iff \forall_{m=1,2,\dots}\colon f_{|D_m}\in L(D)\\
        \text{und}\\
        \exists C > 0\colon \forall_{m=1,2,\dots}\colon\integ{{D_m}}{|{f_{|D_m}(x)}|} < C\\
    \end{align*}
    In diesem Fall gilt: $\lim_{m \to infty}\left(\integ{{D_m}}{{f_{|D_m}}(x)}\right) = \integ{D}{f(x)} \in \mathbb{R}$
\end{korollar}
\subsection{Berechnung von Gebietsintegralen, Satz von Fubini}
Wir haben bereits gesehen: Lebesgue-Integral $\integ{D}{f(x)}$ für $D \subseteq \mathbb{R}$ Intervall, f. ü. stetig ist mit Methoden aus Ana1/2 berechenbar.\\
Nun Höherdimensional\\
\begin{satz}(von Fubini)\\
    Sei $D \subseteq \mathbb{R}^{n+m}$ abg. Quader oder $D = \mathbb{R}^{n+m}$, d.h. $D = D_1 \times D_2$ mit $D_1 \}\subseteq \mathbb{R}^n, \, D_2 \subseteq \mathbb{R}^m$ abg. Quader oder $D_1 = \mathbb{R}^n, \, D_2 = \mathbb{R}^m.$\\
    Ist $\underline{f\in L(D)}, f\colon D_1 \times D_2 \to \mathbb{R} \text{ mit } (x,y) \mapsto f(x,y)$ so existiert eine n-dim Nullmenge $N_1 \subset D_1$ derart, dass für $x \in D_1 \setminus N_1$ die Funktion $f(x,\cdot)\colon D_2 \to \mathbb{R} \text{ mit } y \mapsto f(x,y)$ in $L(D_2)$ liegt und weiter $\exists_{g \in L(D_1)}, \, g\colon D_1 \to \mathbb{R} \text{ mit } x \mapsto g(x)$ für die $g(x) = \int_{D_2} f(x,y) \, \text{d}y$ für alle $x\in D_1\setminus N_1$ gilt, derart, dass \[\underbrace{\int_D f(x,y) \,\text{d}(x,y)}_{\text{Lebesgue-Integral von } f\in L(D)} = \integ{{D_1}}{g(x)} = \int_{D_1} \left( \int_{D_2} f(x,y)\, \text{d}y \right) \, \text{d}x \]
    Analog hat man im gleichen Sinn.
    \[\int_D f(x,y) \, \text{d}(x,y) = \int_{D_2} \left( \int_{D_1} f(x,y) \, \text{d}x\right) \,\text{d}y\]
    \underline{Fazit:} Man kann in diesem Fall $f\in L(D)$ das Integral von f komponentenweise verschachtelt ausrechnen.
\end{satz}

\underline{Beispiel:} $\begin{matrix}
    f\colon & \overbrace{[0,1]\times [2,5]\times [-1,0]}^D &\to& \mathbb{R}\\
    & (x_1,x_2x_3)&\mapsto& x_1 e^{\sin (x_2) + x_3^2}
\end{matrix} \, f \text{ stetig } \Rightarrow f \in L(D)$
\begin{align*}
    \int_D x_1 e^{\sin (x_2) + x_3^2} \, \text{d}(x_1, x_2, x_3) &=& \int_{[-1,0]} \left( \int_{[2,5]} \left(\int_{[0,1]} x_1 e^{\sin (x_2) + x_3^2} \, \text{d}x_1 \right) \, \text{d}x_2 \right) \, \text{d}x_3\\
    &=& \int_{-1}^0 \left(\int_2^5\left(\int_0^1 x_1 e^{\sin (x_2) + x_3^2}\, \text{d}x_1 \right) \, \text{d}x_2 \right) \, \text{d}x_3
\end{align*}

\begin{proof}
    Sei $f\in L(D)$. O. E. $f\in L^\uparrow (D)$ (Allg. Fall $f=g-h \text{ mit } g,h\in L^\uparrow (D)$ folgt unmittelbar) d. h. es existiert eine Folge von Treppenfunktionen $\varphi_n\colon D \to \mathbb{R}$ mit $\varphi_n \to f$ f. ü. mon. wachsend mit
    \begin{align}\label{GW:proof}
        \int_D f(x,y) \, \text{d}(x,y) = \lim_{n \to \infty} \int_D \varphi_n (x,y) \, \text{d}(x,y)
    \end{align}
    Insbesondere existiert eine (m+n)-dim. Nullmenge $N \subseteq D \subseteq \mathbb{R}^{n+m}$ mit $\varphi_n (x,y) \xrightarrow[n \to\infty]{} f(x,y)$ monoton wachsend für alle $(x,y) \in D\setminus N$

    \underline{Hilfssatz:} Ist $N \subseteq D$ (n+m)-dim. Nullmenge, dann ist
    \[\widetilde{N_1} \coloneqq \{x\in D_1\, | \, \underbrace{\{y\in D_2\, |\, (x,y)\in N\}}_{\subseteq D_2} \text{ \underline{keine} n-dim Nullmenge }\} \subseteq D_1\] eine n-dim Nullmenge. Beweis siehe Arens Lemma S. 918

    D. h.
    \begin{align}\label{GW2:proof}
        \forall_{x\in D_1\setminus \widetilde{N_1}} \text{ fest: } \varphi(x,\cdot) \xrightarrow[n\to\infty]{} f(x,\cdot) \text{ f. ü. mon. wachsend}
    \end{align}
    (d. h. außerhalb einer n-dim Nullmenge $\subseteq D_2$)
    
    Offensichtlich gilt $\int_D \varphi_n (x,y) \text{d}(x,y) = \int_{D_1} \psi_n (x) \text{d}x$ mit der Treppenfunktion
    \[\varphi_n \colon D_1 \to \mathbb{R}, \, \psi_n (x) \coloneqq \int_{D_2} \varphi_n (x,y) \text{d}y\]

    Die Folge $(\psi_n)_n$ ist monoton wachsend $\forall_{x\in D_1\setminus \widetilde{N_1}}$ (da die $(\varphi_n)_n$ monoton wachsend für alle $(x,y) \in D \setminus N$)

    Wegen der Beschränktheit in Gleichung \ref{GW:proof} und monotoner Konvergenz existiert
    \begin{align}\label{GW3:proof}
        g\in L(D_1) \text{ mit } \psi_n \xrightarrow[n\to\infty]{} g\in L(D_1) \text{ f. ü. mon. wachsend}
    \end{align}
    mit (wegen Gleichung \ref{GW:proof})
    \[\int_D f(x,y) \, \text{d}(x,y) = \int_{D_1} g(x) \, \text{d}x\\
        \left(=\lim_{n \to \infty} \int_{D_1} \varphi_n (x) \, \text{d}x = \lim_{n\to\infty} \varphi_n (x,y) \, \text{d}(x,y) = \int_D f(x,y) \, \text{d}(x,y)\right)\]
    Bleibt z.z.: $g(x) = \int_{D_2} f(x,y) \, \text{d}y$ für fast alle $x\in D_1$. Wegen \ref{GW3:proof} ist $\underbrace{\int_{D_2} \varphi_n (x,y) \, \text{d}y}_{\psi_n (x)}$ für fast alle $x$ beschränkt und mit $\varphi_n (x,y) \xrightarrow[n\to\infty]{} f(x,y)$ für fast alle $y$ mon. konvergent.
    \begin{align*}
        \Rightarrow \lim_{n\to\infty} \int_{D_2} \varphi_n (x,y) \, \text{d}y &= \int_{D_2} f(x,y) \, \text{d}y \text{ für fast alle } x \in D_1\\
        g(x) = \lim_{n\to\infty} \int_{D_2} \varphi_n (x,y) \, \text{d}y &= \int_{D_2} f(x,y) \, \text{d}y\\
        \Rightarrow \int_D f(x,y) \, \text{d}(x,y) &= \int_{D_1} g(x) \, \text{d}x\\
        &= \int_{D_1} \left(\int_{D_2} f(x,y) \, \text{d}y\right) \, \text{d}x
    \end{align*}
\end{proof}
\underline{Fazit:} Man kann Integrale über Quader ausrechnen.\\
\underline{Achtung:} Der Satz von Fubini sagt nicht, dass eine Funktion $f \colon D \to \mathbb{R}$, wenn $f$ sukzessiv nach $x$ und $y$ integrierbar ist! D. h., wenn z. B. y festgehalten wird, zu sagen, dass $f$ nach $x$ integrierbar ist, reicht nicht aus.\\
Auch die Vertauschbarkeit der Integration ist dann nicht gesichert. Bsp. siehe Arens S. 925
\underline{Beispiele}
\begin{itemize}
    \item[a)] $f\colon \overbrace{[0,1]\times [-1,2] \times [-2,0]}^{\eqqcolon D} \to \mathbb{R} \text{ mit } (x,y,z) \mapsto xy^2 z^3 \, f$ ist stetig (da polynom) $\Rightarrow f\in L(D)$\\
    Fubini \begin{align*}
        \Rightarrow & \int_D xy^2 z^3 \, \text{d}(x,y,z) &=& \int_{[-2,0]}\left(\int_{[-1,2]}\left(\int_{[0,1]} xy^2 z^3 \, \text{d}x \right)\, \text{d}y\right)\, \text{d}z\\
        & &=& \int_{-2}^0 \left(\int_{-1}^2 \left(\int_0^1 x y^2 z^3 \, \text{d}x\right)\, \text{d}y\right)\, \text{d}z\\
        & &=& \int_{-2}^0 \left(\int_1^2 [\frac{1}{2}x^2 y^2 z^3]_{x=0}^{x=1} \, \text{d}y\right)\, \text{d}z\\
        & &=& \int_{-2}^0 \left(\int_1^2 \frac{1}{2}y^2 z^3 \, \text{d}y\right)\, \text{d}z\\
        & &=& \int_{-2}^0 [\frac{1}{6} y^3 z^3]_{y=1}^{y=2} \, \text{d}z\\
        & &=& \int_{-2}^0 \frac{3}{2} z^3 \, \text{d}z = [\frac{3}{8} z^4]_{z=-2}^{z=0} = -\frac{3}{8}(-2)^4 = -6
    \end{align*} 
    \item[b)] \begin{align*}
        f\colon \mathbb{R}^2 &\to \mathbb{R} & \text{ mit } A = \{(x,y) \in \mathbb{R}^2 | 0 \leq x \leq 1, \,x \leq y \leq 1\}\\
        (x,y) &\mapsto \left\{\begin{matrix}
            xy^2 ,& (x,y) \in A\\
            0 ,& \text{ sonst}
        \end{matrix}\right. & = \{(x,y)\in\mathbb{R}^2 | 0\leq y\leq 1, \,0\leq x\leq y\}
    \end{align*}
    Mit \ref{Intbar:Lebesgue} folgt, dass $f\in L(\mathbb{R}^2)$, da nur an den Randpunkten des Dreiecks unstetig, das ist aber eine Nullmenge.\\
    Mit \(f(x,y) = \id_A (x,y)\, xy^2\), Indikatorfunktion \(\id_A (x,y) \coloneqq \left\{ \begin{matrix}
        1,& (x,y) \in A\\
        0,& (x,y) \notin A
    \end{matrix} \right.\)
    \begin{align*}
        \int_{\mathbb{R}^2} f(x,y) \,\text{d}(x,y) &= \int_{\mathbb{R}^2} \id_A (x,y) \, xy^2 \, \text{d}(x,y) & \\
        &= \int_{\mathbb{R}} \left(\int_{\mathbb{R}} \id_A (x,y)\, xy^2 \, \text{d}y \right)\, \text{d}x & \text{(Fubini)} \\
        &= \int_{\mathbb{R}} \id_{[0,1]}(x) \left(\int_{\mathbb{R}} \id_{[x,1]}(y) \, xy^2 \, \text{d}y \right) \, \text{d}x & \\
        &= \int_{\mathbb{R}} \id_{[0,1]}(x) \left(\int_x^1 xy^2 \, \text{d}y \right) \,\text{d}x & \\
        &= \int_{\mathbb{R}} \id_{[0,1]}(x)\, \left[\frac{x}{3}y^3\right]_{y=x}^{y=1} \,\text{d}x & \\
        &= \int_0^1 (\frac{x}{3} - \frac{x^4}{3}) \,\text{d}x & \\
        &= \left[\frac{x^2}{6} - \frac{x^5}{15}\right]_{x=0}^{x=1} = \frac{1}{10} &
    \end{align*}
\end{itemize}
\underline{Allgemein:} (Praktische Berechnung) Normalbereiche
Sei $A \subseteq \mathbb{R}^2$ mit $A= \{(x,y) \in \mathbb{R}^2 \, | \, a \leq x \leq b, \, \alpha (x)\leq y \leq \beta (x)\}$, wobei $a\leq b$ und $\alpha ,\beta \colon [a,b] \to \mathbb{R}$ stetig mit $\alpha (x) \leq \beta (x) \forall_{x\in[a,b]}$\\
$f\colon \mathbb{R}^2 \to \mathbb{R}$ sei stetig auf $A$ und $O$ auf $\mathbb{R}^2\setminus A$, dann gilt mit Fubini:
\begin{equation*}
    \int_{\mathbb{R}^2} f(x,y) \,\text{d}(x,y) = \int_a^b \left(\int_{\alpha (x)}^{\beta (x)} f(x,y) \,\text{d}y\right)\,\text{d}x \eqqcolon \int_A f(x,y) \, \text{d}(x,y)
\end{equation*}
Ein $A \subseteq \mathbb{R}^2$ obiger Gestalt nennt man einen Normalbereich in $x$. Gibt es analog $c\leq d, \gamma, \delta\colon [c,d]\to \mathbb{R}$ stetig, $\gamma (y) \leq \delta (y)\forall y\in [c,d]$ mit $A=\{(x,y) \in\mathbb{R}^2\,|\,c\leq y\leq d,\,\gamma (y) \leq x\leq \delta (y)\}$ so nennt man $A$ Normalbereich in $y$. Damit analog
\begin{equation*}
    \int_A f(x,y) \,\text{d}(x,y) = \int_c^d \left(\int_{\gamma (y)}^{\delta(y)} f(x,y)\,\text{d}x\right)\,\text{d}y
\end{equation*}

Nun wollen wir für $D$ eine allgemeinere Gestalt als abgeschlossene Quader bzw. $\mathbb{R}^n$ erreichen.

\begin{definition}\label{def:messbar}
    \begin{itemize}
        \item[a)] Eine Funktion $f\colon \mathbb{R}^n\to\mathbb{R}$ heißt \underline{messbar}, falls eine Folge von Treppenfunktionen $\varphi_n\colon\mathbb{R}^n\to\mathbb{R}$ ex. mit $\varphi_n \xrightarrow[n\to\infty]{} f$ f. ü.
        \item[b)] Eine Menge $A\subseteq \mathbb{R}^n$ heißt \underline{messbar}, falls die Indikatorfunktion \begin{align*}
            \id_A\colon & \mathbb{R}^n\to\mathbb{R} & \text{ messbar ist.}\\
            &x\mapsto\left\{\begin{matrix*}
                1,&x\in A\\
                0,&x\notin A
            \end{matrix*}\right. &
        \end{align*}
    \end{itemize}
\end{definition}
\underline{Bemerkung} Die Definition \ref{def:messbar} ist allgemeiner als für $f\in L^{\uparrow}(D),\, D=\mathbb{R}^n$, da $(\varphi_n)_n$ monoton wachsend f. ü. nicht verlangt wird!

\begin{lemma}
    \begin{itemize}
        \item[a)] Jede offene Teilmenge $U\subseteq\mathbb{R}^n$ ist messbar.
        \item[b)] Jede abgeschlossene Teilmenge $A\subseteq\mathbb{R}^n$ ist messbar. 
    \end{itemize}
\end{lemma}
\begin{proof}
    \begin{itemize}
        \item[Ad a)] Sei $U\subseteq\mathbb{R}^n$ offen. Für $x\in U$ definiere $\tilde{\text{d}}(x) \coloneqq \text{sup}\{r>0|\text{B}(x)\subseteq U\}\\
        \text{d}(x) \coloneqq \left\{\begin{matrix}
            \tilde{\text{d}}(x), & \tilde{\text{d}}(x)\leq 1\\
            1, & \text{ sonst}
        \end{matrix}\right.$\\
        Sei $U \cap \mathbb{Q}^n =\{x_1,x_2,\dots\}$, definier induktiv Treppenfunktionen: \begin{equation*}
            \varphi_1\coloneqq \id_{Q_{x_1, \,\frac{\text{d}(x_1)}{\sqrt{n}}}}
        \end{equation*} mit $Q_{y,\epsilon}\coloneqq$ abg. Quader mit Seitenlänge $\epsilon$ und Mittelpunkt $y$, also $Q_{x_1, \,\frac{\text{d}(x_1)}{\sqrt{n}}} \subseteq U$ (Insbesondere $\varphi_1 \leq \id_U$)\\
        $\varphi_{k+1}\coloneqq\text{max}(\varphi_k,\, \id_{Q_{x_{k+1}, \,\frac{\text{d}(x_{k+1})}{\sqrt{n}}}}) k=1,2,\dots$. $\varphi_k$ sind alle Treppenfunktionen mit $\varphi_k \to \id_U$, denn $\forall_{x\in U} \,\exists_{0<\epsilon<1}$ mit $\text{B}_{\epsilon} (x) \subseteq U$. $\mathbb{Q}^n$ liegt dicht in $\mathbb{R}^n$, d. h. \[\exists_{\tilde{x}\in\mathbb{Q}^n}\colon ||x-\tilde{x}|| < \frac{\epsilon}{4}<\epsilon\]
        Daraus folgt $\tilde{x}\in U\cap\mathbb{Q}^n$, d. h. $\tilde{x}=x_k$ für ein geignetes $k\in\mathbb{N}$ und $\text{d}(x_k) > \frac{\epsilon}{2}$, d. h. $x\in Q_{x_k, \, \frac{\text{d}(x_k)}{\sqrt{n}}}$, d. h. $\varphi_i(x)= 1 \forall_{i \geq k}$, d. h. $\varphi_i(x)\xrightarrow[i\to\infty]{}\id_U(x)$\\
        Andererseits:\\
        Für $x\in\mathbb{R}^n\setminus U \colon x\notin Q_{x_k, \, \frac{\text{d}(x_k)}{\sqrt{n}}} \forall_{k\in\mathbb{N}}$, d. h.\\ $\varphi_k(x) = 0 \, \forall_{k\in\mathbb{N}}$, d. h. $\varphi_i(x))\xrightarrow[i\to\infty]{}\id_U(x)$
        \item[Ad b)] $U \coloneqq \mathbb{R}^n \setminus A$ offen. Mit den $\varphi_k$ aus a) definiere $\psi_k \coloneqq \id_{[-k,k]^n}-\varphi_k$.\\
        Dann gilt: $\forall_{x\in\mathbb{R}^n}\colon \psi_k (x) \xrightarrow[i\to\infty]{}\id_A(x)$
    \end{itemize}
\end{proof}

\end{document}
