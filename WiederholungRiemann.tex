\subsubsection{Wiederholung Riemann-Integral}
Eine Funktion $f \colon [a,b] \to \mathbb{R}$ heißt Riemann-integrierbar, falls es eine Folge von Treppenfunktionen $\varphi_n, \psi_n\colon [a,b] \to \mathbb{R} \, ( n = 1,2,3,...) \text{ mit }$\\ \[
\varphi_n \leq f \leq \psi_n \text{ und } \lim_{n \to \infty} \int_{a}^{b} \varphi_n(x) \,\text{d}x = \lim_{n \to \infty} \int_{a}^{b} \psi_n(x) \, \text{d}x = J\]
gibt.\\
Man definiert dann $\int_{a}^{b} f(x) \text{d}x = J$. Eine Treppenfunktion beschreibt dabei eine Abbildung folgender Art:
\[\varphi\colon [a,b] \to \mathbb{R} \text{, mit } \exists a = x_0<x_1<x_2<...<x_n=b \] derart, dass für alle $k=1,2,...,n \, \varphi$ konstant auf $]x_{k-1}, x_k[$ ist.\\
$\int_{a}^{b}\varphi(x) \text{d}x = \sum_{k=1}^{n} \varphi(\xi_k)(x_k - x_{k-1})$ mit $\xi_k \in ]x_{k-1}, x_1[$ beliebig.\\
Diesen Integralsbegriff werden wir zu den Lebesgue-Integralen erweitern.\\
\underline{Bemerkung:} Es gilt $f\colon [a,b] \to \mathbb{R}$ ist Riemann-integrierbar genau dann, wenn die Menge $U \subset [a,b]$ der Unstetigkeitsmengen von $f$ eine Nullmenge und f beschränkt ist.\\
Dazu zwei Beispiele:\\
\\
Dirichletsche Sprungfunktion:\\
$\begin{matrix}
		f\colon [a,b] & \to & \mathbb{R}\\
		x & \mapsto & \left \{
		\begin{matrix}
			1 & x \text{ rational}\\
			0 & \text{sonst}
		\end{matrix}
		\right .
\end{matrix}$\\
$U = [0,1]$ ist keine Nullmenge, also $f$ nicht Riemann-integrierbar.\\

Die Thomae-Funktion:\\
$\begin{matrix}
	f\colon [0,1] & \to & \mathbb{R}\\
	x & \mapsto & \left \{
	\begin{matrix}
		\frac{1}{q} & x = \frac{p}{q} \, p,q \in \mathbb{N}, ggT(p,q)=1\\
		0 & x \in \mathbb{Q}
	\end{matrix}
	\right .
\end{matrix}$\\
Hier gilt: $U = [0,1] \cap \mathbb{Q}$

