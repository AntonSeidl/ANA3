\section*{Einführung}
Dieses Skript basiert auf einer Mitschrift der Vorlesung Analysis $\RM{3}$, des Wintersemesters 18/19 an der OTH - Regensburg, gehalten von Herrn Prof. Dr. Illies. Das allgemeine Ziel der Vorlesung ist die Integration im $\mathbb{R}^n$, genauer Gebiets-, Kurven- und Oberflächenintegrale, sowie die dazugehörenden Integrationssätze von Gauß, Stokes und Green.\\
Um dies zu erreichen werden wir das Riemannintegral im $\mathbb{R}^1$ zu dem Lebesgueintegral verallgemeinern. Dabei stellen wir fest, dass das Riemannintegral ein Spezialfall des Lebesgueintegrals ist und mit den Sätzen der majorisierten und der monotonen Konvergenz, dass zweiteres bessere Konvergenzeigenschaften hat.