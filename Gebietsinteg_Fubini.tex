\subsection{Berechnung von Gebietsintegralen, Satz von Fubini}
Wir haben bereits gesehen: Lebesgue-Integral $\integ{D}{f(x)}$ für $D \subseteq \mathbb{R}$ Intervall, f. ü. stetig ist mit Methoden aus Ana1/2 berechenbar.\\
Nun Höherdimensional\\
\begin{satz}(von Fubini)\\
    Sei $D \subseteq \mathbb{R}^{n+m}$ abg. Quader oder $D = \mathbb{R}^{n+m}$, d.h. $D = D_1 \times D_2$ mit $D_1 \}\subseteq \mathbb{R}^n, \, D_2 \subseteq \mathbb{R}^m$ abg. Quader oder $D_1 = \mathbb{R}^n, \, D_2 = \mathbb{R}^m.$\\
    Ist $\underline{f\in L(D)}, f\colon D_1 \times D_2 \to \mathbb{R} \text{ mit } (x,y) \mapsto f(x,y)$ so existiert eine n-dim Nullmenge $N_1 \subset D_1$ derart, dass für $x \in D_1 \setminus N_1$ die Funktion $f(x,\cdot)\colon D_2 \to \mathbb{R} \text{ mit } y \mapsto f(x,y)$ in $L(D_2)$ liegt und weiter $\exists_{g \in L(D_1)}, \, g\colon D_1 \to \mathbb{R} \text{ mit } x \mapsto g(x)$ für die $g(x) = \int_{D_2} f(x,y) \, \text{d}y$ für alle $x\in D_1\setminus N_1$ gilt, derart, dass \[\underbrace{\int_D f(x,y) \,\text{d}(x,y)}_{\text{Lebesgue-Integral von } f\in L(D)} = \integ{{D_1}}{g(x)} = \int_{D_1} \left( \int_{D_2} f(x,y)\, \text{d}y \right) \, \text{d}x \]
    Analog hat man im gleichen Sinn.
    \[\int_D f(x,y) \, \text{d}(x,y) = \int_{D_2} \left( \int_{D_1} f(x,y) \, \text{d}x\right) \,\text{d}y\]
    \underline{Fazit:} Man kann in diesem Fall $f\in L(D)$ das Integral von f komponentenweise verschachtelt ausrechnen.
\end{satz}

\underline{Beispiel:} $\begin{matrix}
    f\colon & \overbrace{[0,1]\times [2,5]\times [-1,0]}^D &\to& \mathbb{R}\\
    & (x_1,x_2x_3)&\mapsto& x_1 e^{\sin (x_2) + x_3^2}
\end{matrix} \, f \text{ stetig } \Rightarrow f \in L(D)$
\begin{align*}
    \int_D x_1 e^{\sin (x_2) + x_3^2} \, \text{d}(x_1, x_2, x_3) &=& \int_{[-1,0]} \left( \int_{[2,5]} \left(\int_{[0,1]} x_1 e^{\sin (x_2) + x_3^2} \, \text{d}x_1 \right) \, \text{d}x_2 \right) \, \text{d}x_3\\
    &=& \int_{-1}^0 \left(\int_2^5\left(\int_0^1 x_1 e^{\sin (x_2) + x_3^2}\, \text{d}x_1 \right) \, \text{d}x_2 \right) \, \text{d}x_3
\end{align*}

\begin{proof}
    Sei $f\in L(D)$. O. E. $f\in L^\uparrow (D)$ (Allg. Fall $f=g-h \text{ mit } g,h\in L^\uparrow (D)$ folgt unmittelbar) d. h. es existiert eine Folge von Treppenfunktionen $\varphi_n\colon D \to \mathbb{R}$ mit $\varphi_n \to f$ f. ü. mon. wachsend mit
    \begin{align}\label{GW:proof}
        \int_D f(x,y) \, \text{d}(x,y) = \lim_{n \to \infty} \int_D \varphi_n (x,y) \, \text{d}(x,y)
    \end{align}
    Insbesondere existiert eine (m+n)-dim. Nullmenge $N \subseteq D \subseteq \mathbb{R}^{n+m}$ mit $\varphi_n (x,y) \xrightarrow[n \to\infty]{} f(x,y)$ monoton wachsend für alle $(x,y) \in D\setminus N$

    \underline{Hilfssatz:} Ist $N \subseteq D$ (n+m)-dim. Nullmenge, dann ist
    \[\widetilde{N_1} \coloneqq \{x\in D_1\, | \, \underbrace{\{y\in D_2\, |\, (x,y)\in N\}}_{\subseteq D_2} \text{ \underline{keine} n-dim Nullmenge }\} \subseteq D_1\] eine n-dim Nullmenge. Beweis siehe Arens Lemma S. 918

    D. h.
    \begin{align}\label{GW2:proof}
        \forall_{x\in D_1\setminus \widetilde{N_1}} \text{ fest: } \varphi(x,\cdot) \xrightarrow[n\to\infty]{} f(x,\cdot) \text{ f. ü. mon. wachsend}
    \end{align}
    (d. h. außerhalb einer n-dim Nullmenge $\subseteq D_2$)
    
    Offensichtlich gilt $\int_D \varphi_n (x,y) \text{d}(x,y) = \int_{D_1} \psi_n (x) \text{d}x$ mit der Treppenfunktion
    \[\varphi_n \colon D_1 \to \mathbb{R}, \, \psi_n (x) \coloneqq \int_{D_2} \varphi_n (x,y) \text{d}y\]

    Die Folge $(\psi_n)_n$ ist monoton wachsend $\forall_{x\in D_1\setminus \widetilde{N_1}}$ (da die $(\varphi_n)_n$ monoton wachsend für alle $(x,y) \in D \setminus N$)

    Wegen der Beschränktheit in Gleichung \ref{GW:proof} und monotoner Konvergenz existiert
    \begin{align}\label{GW3:proof}
        g\in L(D_1) \text{ mit } \psi_n \xrightarrow[n\to\infty]{} g\in L(D_1) \text{ f. ü. mon. wachsend}
    \end{align}
    mit (wegen Gleichung \ref{GW:proof})
    \[\int_D f(x,y) \, \text{d}(x,y) = \int_{D_1} g(x) \, \text{d}x\\
        \left(=\lim_{n \to \infty} \int_{D_1} \varphi_n (x) \, \text{d}x = \lim_{n\to\infty} \varphi_n (x,y) \, \text{d}(x,y) = \int_D f(x,y) \, \text{d}(x,y)\right)\]
    Bleibt z.z.: $g(x) = \int_{D_2} f(x,y) \, \text{d}y$ für fast alle $x\in D_1$. Wegen \ref{GW3:proof} ist $\underbrace{\int_{D_2} \varphi_n (x,y) \, \text{d}y}_{\psi_n (x)}$ für fast alle $x$ beschränkt und mit $\varphi_n (x,y) \xrightarrow[n\to\infty]{} f(x,y)$ für fast alle $y$ mon. konvergent.
    \begin{align*}
        \Rightarrow \lim_{n\to\infty} \int_{D_2} \varphi_n (x,y) \, \text{d}y &= \int_{D_2} f(x,y) \, \text{d}y \text{ für fast alle } x \in D_1\\
        g(x) = \lim_{n\to\infty} \int_{D_2} \varphi_n (x,y) \, \text{d}y &= \int_{D_2} f(x,y) \, \text{d}y\\
        \Rightarrow \int_D f(x,y) \, \text{d}(x,y) &= \int_{D_1} g(x) \, \text{d}x\\
        &= \int_{D_1} \left(\int_{D_2} f(x,y) \, \text{d}y\right) \, \text{d}x
    \end{align*}
\end{proof}
\underline{Fazit:} Man kann Integrale über Quader ausrechnen.\\
\underline{Achtung:} Der Satz von Fubini sagt nicht, dass eine Funktion $f \colon D \to \mathbb{R}$, wenn $f$ sukzessiv nach $x$ und $y$ integrierbar ist! D. h., wenn z. B. y festgehalten wird, zu sagen, dass $f$ nach $x$ integrierbar ist, reicht nicht aus.\\
Auch die Vertauschbarkeit der Integration ist dann nicht gesichert. Bsp. siehe Arens S. 925
\underline{Beispiele}
\begin{itemize}
    \item[a)] $f\colon \overbrace{[0,1]\times [-1,2] \times [-2,0]}^{\eqqcolon D} \to \mathbb{R} \text{ mit } (x,y,z) \mapsto xy^2 z^3 \, f$ ist stetig (da polynom) $\Rightarrow f\in L(D)$\\
    Fubini \begin{align*}
        \Rightarrow & \int_D xy^2 z^3 \, \text{d}(x,y,z) &=& \int_{[-2,0]}\left(\int_{[-1,2]}\left(\int_{[0,1]} xy^2 z^3 \, \text{d}x \right)\, \text{d}y\right)\, \text{d}z\\
        & &=& \int_{-2}^0 \left(\int_{-1}^2 \left(\int_0^1 x y^2 z^3 \, \text{d}x\right)\, \text{d}y\right)\, \text{d}z\\
        & &=& \int_{-2}^0 \left(\int_1^2 [\frac{1}{2}x^2 y^2 z^3]_{x=0}^{x=1} \, \text{d}y\right)\, \text{d}z\\
        & &=& \int_{-2}^0 \left(\int_1^2 \frac{1}{2}y^2 z^3 \, \text{d}y\right)\, \text{d}z\\
        & &=& \int_{-2}^0 [\frac{1}{6} y^3 z^3]_{y=1}^{y=2} \, \text{d}z\\
        & &=& \int_{-2}^0 \frac{3}{2} z^3 \, \text{d}z = [\frac{3}{8} z^4]_{z=-2}^{z=0} = -\frac{3}{8}(-2)^4 = -6
    \end{align*} 
    \item[b)] \begin{align*}
        f\colon \mathbb{R}^2 &\to \mathbb{R} & \text{ mit } A = \{(x,y) \in \mathbb{R}^2 | 0 \leq x \leq 1, \,x \leq y \leq 1\}\\
        (x,y) &\mapsto \left\{\begin{matrix}
            xy^2 ,& (x,y) \in A\\
            0 ,& \text{ sonst}
        \end{matrix}\right. & = \{(x,y)\in\mathbb{R}^2 | 0\leq y\leq 1, \,0\leq x\leq y\}
    \end{align*}
    Mit \ref{Intbar:Lebesgue} folgt, dass $f\in L(\mathbb{R}^2)$, da nur an den Randpunkten des Dreiecks unstetig, das ist aber eine Nullmenge.\\
    Mit \(f(x,y) = \id_A (x,y)\, xy^2\), Indikatorfunktion \(\id_A (x,y) \coloneqq \left\{ \begin{matrix}
        1,& (x,y) \in A\\
        0,& (x,y) \notin A
    \end{matrix} \right.\)
    \begin{align*}
        \int_{\mathbb{R}^2} f(x,y) \,\text{d}(x,y) &= \int_{\mathbb{R}^2} \id_A (x,y) \, xy^2 \, \text{d}(x,y) & \\
        &= \int_{\mathbb{R}} \left(\int_{\mathbb{R}} \id_A (x,y)\, xy^2 \, \text{d}y \right)\, \text{d}x & \text{(Fubini)} \\
        &= \int_{\mathbb{R}} \id_{[0,1]}(x) \left(\int_{\mathbb{R}} \id_{[x,1]}(y) \, xy^2 \, \text{d}y \right) \, \text{d}x & \\
        &= \int_{\mathbb{R}} \id_{[0,1]}(x) \left(\int_x^1 xy^2 \, \text{d}y \right) \,\text{d}x & \\
        &= \int_{\mathbb{R}} \id_{[0,1]}(x)\, \left[\frac{x}{3}y^3\right]_{y=x}^{y=1} \,\text{d}x & \\
        &= \int_0^1 (\frac{x}{3} - \frac{x^4}{3}) \,\text{d}x & \\
        &= \left[\frac{x^2}{6} - \frac{x^5}{15}\right]_{x=0}^{x=1} = \frac{1}{10} &
    \end{align*}
\end{itemize}
\underline{Allgemein:} (Praktische Berechnung) Normalbereiche
Sei $A \subseteq \mathbb{R}^2$ mit $A= \{(x,y) \in \mathbb{R}^2 \, | \, a \leq x \leq b, \, \alpha (x)\leq y \leq \beta (x)\}$, wobei $a\leq b$ und $\alpha ,\beta \colon [a,b] \to \mathbb{R}$ stetig mit $\alpha (x) \leq \beta (x) \forall_{x\in[a,b]}$\\
$f\colon \mathbb{R}^2 \to \mathbb{R}$ sei stetig auf $A$ und $O$ auf $\mathbb{R}^2\setminus A$, dann gilt mit Fubini:
\begin{equation*}
    \int_{\mathbb{R}^2} f(x,y) \,\text{d}(x,y) = \int_a^b \left(\int_{\alpha (x)}^{\beta (x)} f(x,y) \,\text{d}y\right)\,\text{d}x \eqqcolon \int_A f(x,y) \, \text{d}(x,y)
\end{equation*}
Ein $A \subseteq \mathbb{R}^2$ obiger Gestalt nennt man einen Normalbereich in $x$. Gibt es analog $c\leq d, \gamma, \delta\colon [c,d]\to \mathbb{R}$ stetig, $\gamma (y) \leq \delta (y)\forall y\in [c,d]$ mit $A=\{(x,y) \in\mathbb{R}^2\,|\,c\leq y\leq d,\,\gamma (y) \leq x\leq \delta (y)\}$ so nennt man $A$ Normalbereich in $y$. Damit analog
\begin{equation*}
    \int_A f(x,y) \,\text{d}(x,y) = \int_c^d \left(\int_{\gamma (y)}^{\delta(y)} f(x,y)\,\text{d}x\right)\,\text{d}y
\end{equation*}

Nun wollen wir für $D$ eine allgemeinere Gestalt als abgeschlossene Quader bzw. $\mathbb{R}^n$ erreichen.

\begin{definition}\label{def:messbar}
    \begin{itemize}
        \item[a)] Eine Funktion $f\colon \mathbb{R}^n\to\mathbb{R}$ heißt \underline{messbar}, falls eine Folge von Treppenfunktionen $\varphi_n\colon\mathbb{R}^n\to\mathbb{R}$ ex. mit $\varphi_n \xrightarrow[n\to\infty]{} f$ f. ü.
        \item[b)] Eine Menge $A\subseteq \mathbb{R}^n$ heißt \underline{messbar}, falls die Indikatorfunktion \begin{align*}
            \id_A\colon & \mathbb{R}^n\to\mathbb{R} & \text{ messbar ist.}\\
            &x\mapsto\left\{\begin{matrix*}
                1,&x\in A\\
                0,&x\notin A
            \end{matrix*}\right. &
        \end{align*}
    \end{itemize}
\end{definition}
\underline{Bemerkung} Die Definition \ref{def:messbar} ist allgemeiner als für $f\in L^{\uparrow}(D),\, D=\mathbb{R}^n$, da $(\varphi_n)_n$ monoton wachsend f. ü. nicht verlangt wird!

\begin{lemma}\label{Lem:Messbar}
    \begin{itemize}
        \item[a)] Jede offene Teilmenge $U\subseteq\mathbb{R}^n$ ist messbar.
        \item[b)] Jede abgeschlossene Teilmenge $A\subseteq\mathbb{R}^n$ ist messbar. 
    \end{itemize}
\end{lemma}
\begin{proof}
    \begin{itemize}
        \item[Ad a)] Sei $U\subseteq\mathbb{R}^n$ offen. Für $x\in U$ definiere $\tilde{\text{d}}(x) \coloneqq \text{sup}\{r>0|\text{B}(x)\subseteq U\}\\
        \text{d}(x) \coloneqq \left\{\begin{matrix}
            \tilde{\text{d}}(x), & \tilde{\text{d}}(x)\leq 1\\
            1, & \text{ sonst}
        \end{matrix}\right.$\\
        Sei $U \cap \mathbb{Q}^n =\{x_1,x_2,\dots\}$, definier induktiv Treppenfunktionen: \begin{equation*}
            \varphi_1\coloneqq \id_{Q_{x_1, \,\frac{\text{d}(x_1)}{\sqrt{n}}}}
        \end{equation*} mit $Q_{y,\epsilon}\coloneqq$ abg. Quader mit Seitenlänge $\epsilon$ und Mittelpunkt $y$, also $Q_{x_1, \,\frac{\text{d}(x_1)}{\sqrt{n}}} \subseteq U$ (Insbesondere $\varphi_1 \leq \id_U$)\\
        $\varphi_{k+1}\coloneqq\text{max}(\varphi_k,\, \id_{Q_{x_{k+1}, \,\frac{\text{d}(x_{k+1})}{\sqrt{n}}}}) k=1,2,\dots$. $\varphi_k$ sind alle Treppenfunktionen mit $\varphi_k \to \id_U$, denn $\forall_{x\in U} \,\exists_{0<\epsilon<1}$ mit $\text{B}_{\epsilon} (x) \subseteq U$. $\mathbb{Q}^n$ liegt dicht in $\mathbb{R}^n$, d. h. \[\exists_{\tilde{x}\in\mathbb{Q}^n}\colon ||x-\tilde{x}|| < \frac{\epsilon}{4}<\epsilon\]
        Daraus folgt $\tilde{x}\in U\cap\mathbb{Q}^n$, d. h. $\tilde{x}=x_k$ für ein geignetes $k\in\mathbb{N}$ und $\text{d}(x_k) > \frac{\epsilon}{2}$, d. h. $x\in Q_{x_k, \, \frac{\text{d}(x_k)}{\sqrt{n}}}$, d. h. $\varphi_i(x)= 1 \forall_{i \geq k}$, d. h. $\varphi_i(x)\xrightarrow[i\to\infty]{}\id_U(x)$\\
        Andererseits:\\
        Für $x\in\mathbb{R}^n\setminus U \colon x\notin Q_{x_k, \, \frac{\text{d}(x_k)}{\sqrt{n}}} \forall_{k\in\mathbb{N}}$, d. h.\\ $\varphi_k(x) = 0 \, \forall_{k\in\mathbb{N}}$, d. h. $\varphi_i(x))\xrightarrow[i\to\infty]{}\id_U(x)$
        \item[Ad b)] $U \coloneqq \mathbb{R}^n \setminus A$ offen. Mit den $\varphi_k$ aus a) definiere $\psi_k \coloneqq \id_{[-k,k]^n}-\varphi_k$.\\
        Dann gilt: $\forall_{x\in\mathbb{R}^n}\colon \psi_k (x) \xrightarrow[i\to\infty]{}\id_A(x)$
    \end{itemize}
\end{proof}

\underline{Beispiele:}\\
Alle $d\in L(\mathbb{R}^n)$ sind messbar:
\begin{itemize}
    \item $f=g-h$ mit $g,h\in L^\uparrow(\mathbb{R}^n)$, das heißt es existieren Treppenfunktionen $\varphi_k, \psi_k$ mit $\varphi_k \xrightarrow[k\to\infty]{} g, \psi_k \xrightarrow[k\to \infty]{} h$ jeweils fast überall, also $\underbrace{(\varphi_k - \psi_k)}_{\text{Treppenfunktion}} \xrightarrow[k\to \infty]{} f$ fast überall.\\
    Die Umkehrung gilt nur „fast“, es scheitert daran, dass die Integrale unbeschränkt sein können, zum Beispiel:\\
    $f\colon \mathbb{R}\to\mathbb{R}, x\mapsto |x|$ ist messbar: \begin{align*}
        \varphi_k\colon & \mathbb{R} \to \mathbb{R}\\
        & x \mapsto \left\{\begin{matrix}
            \frac{1}{k} [k|x|], \, |x| \leq k\\
            0, \text{ sonst}
        \end{matrix}\right.
    \end{align*}
    also $\varphi_k\to f$. Aber $f\notin L(\mathbb{R})$, da $\int_{-\infty}^{\infty} f(x) \, \text{d}x = \infty$
\end{itemize}

\begin{definition}
    Ist $A\subseteq \mathbb{R}^n$ messbar, so heißt \\
    $\mu (A)\coloneqq \left\{\begin{matrix*}
        \int_{\mathbb{R}^n} \id_A (x) \, \text{d}x, \text{ falls existent}\\
        \infty, \text{ sonst}
    \end{matrix*}\right.$
    \underline{Lebesgue-Borel-Maß} von $A$.

    \underline{Beachte:} $A\subseteq \mathbb{R}^n$ messbar $\Rightarrow \forall \text{ abg. Quader } D_i \subseteq \mathbb{R}^n\colon \id_{A\cap D_i}\in L(D_i)$, das heißt $\mu (A \cap D_i) < \infty$ (Dazu $(\mu (A \cap D_i))_i$ ist monoton wachsend)\\
    Falls $D_1 \subseteq D_2 \subseteq D_3 \subseteq \dots$ und $\bigcup_{i=1}^{\infty} D_i = \mathbb{R}^n$, dann (weil monoton konvergent) folgt $\mu (A) = lim_{j\to\infty} \mu (A\cap D_j)$
\end{definition}

\begin{remark}
    Normalbereiche im $\mathbb{R}^n$ sind messbar (da abgeschlossen), $\mu (A)$ entspricht dem Flächeninhalt von $A$.
    \begin{itemize}
        \item die $\emptyset$ ist messbar mit Treppenfunktion ist die Nullfunktion
        \item Ü. 14 b) nicht jede Menge besitzt Volumina
    \end{itemize}
\end{remark}

\begin{definition}
    Sei $D\subseteq\mathbb{R}^n$ messbar und $f\colon D\to \mathbb{R}$ eine Funktion, dann def. $f\in L(D)\colon \Leftrightarrow$
        \begin{itemize}
            \item[1)]
            \begin{align*}
                \hat{f}\colon & \mathbb{R}^n \to \mathbb{R} \text{ ist messbar}\\
                & x \mapsto \left\{
                    \begin{matrix*}
                    f(x), x \in D\\
                    0, \text{ sonst}
                    \end{matrix*}\right.
            \end{align*}
            \item[2)] $\hat{f} \in L(\mathbb{R}^n)$\\
            Dann $\integ{D}{f(x)} \coloneqq \integ{{\mathbb{R}^n}}{{\hat{f}(x)}}$ 
        \end{itemize}
    Das verallgemeinert die Definition von Lebesgue-Integralen auf beliebige messbare Definitionsbereiche $D\subseteq\mathbb{R}^n$ statt abg. Quader $D\subset\mathbb{R}^n$ oder $D=\mathbb{R}^n$.\\
    Die Definition ist offenbar verträglich mit der Definition aus 1.2 REFERENZ EINFÜGEN!\\
    Alle im Abschnitt 1.3 und 1.4 REFERENZ! bewiesenen Aussagen für L(D) gelten analog für diese Verallgemeinerung, insbesondere monotone Konvergenz sowie majorisierte Konvergenz usw.\\
    Diese Definition ist verträglich mit der Definition $L(D)$ für $D\subseteq\mathbb{R}^1$ abgeschlossener Quader oder $D=\mathbb{R}^1$ und auch mit der Definition $\integ{D}{f(x)}$ für Normalbereiche $D\subseteq\mathbb{R}^2$
\end{definition}

\begin{satz}
    Ist $K\subseteq\mathbb{R}^n$ kompakt und $f\colon K\to\mathbb{R}$ stetig, dann $f\in L(K)$.
\end{satz}

\begin{proof}
        Zunächst ist $K$ abgeschlossen, also messbar (\ref{Lem:Messbar}). $f$ ist stetig auf Kompaktum, damit ist $f$ beschränkt.\\
        Übung 11 a) $\Rightarrow$ es reicht zu zeigen: \begin{align*}
            \hat{f}\colon \mathbb{R}^n\to\mathbb{R}\\
            x\mapsto\left\{\begin{matrix*}
                f(x), x\in K\\
                0, \text{ sonst}
            \end{matrix*}\right.
        \end{align*}
        (denn dann ist 1. erfüllt und weiter: $\hat{f}$ beschränkt. $\Rightarrow \forall \text{ abg. Quader } D\colon f_{|D}\in L(D)$. wähle abg. Quader $D\subseteq\mathbb{R}^n$ mit $K\subseteq D \Rightarrow \hat{f}(x) = 0 \forall_{x\notin D} \Rrightarrow \hat{f}\in L(\mathbb{R}^n))$\\
        Sei $K\subseteq Q$ mit $Q$ offener Quader, $Q \subseteq \mathbb{R}^n$. Für $k\in\mathbb{N}$ zerlege $Q$ disjunkt in $k^n$ Teilquader $Q_1^{(k)}, Q_2^{(k)}, \dots , Q_{k^n}^{(k)}$ deren Seitenlängen genau $\frac{1}{k} \cdot$ Seitenlängen von Q seien. (Die Randpunkte werden geeignet verteilt)\\
        Definiere \begin{align*}
            \tilde{\varphi_k}\colon \mathbb{R}^n\to&\mathbb{R}\\
            & x\mapsto\left\{\begin{matrix*}
                \alpha_i^{(k)}, falls x\in Q_i^{(k)} \text{ mit } Q_i^{(k)} \cap K \neq \emptyset\\
                0, \text{ sonst}
            \end{matrix*}\right.
        \end{align*} wobei $\alpha_i^{(k)} = f(x_i^{(k)})$ mit einem für den Quader $Q_i^{(k)}$ fest gewählten $x_i^{(k)} \in Q_i^{(k)} \cap K$\\
        Die $\tilde{\varphi}_n$ sind Treppenfunktionen und es gilt wegen der Stetigkeit von $f: \tilde{\varphi}_k(x) \xrightarrow[k\to\infty]{} \hat{f}(x) \forall_{x\in\mathbb{R}^n}$
\end{proof}

\begin{satz}
    (Cavalieri-Prinzip)\\
    Ist $A\subseteq\mathbb{R}^{n+1}$ messbar mit $A\subseteq[a,b]\times\mathbb{R}^n$ und $\mu (A) < \infty$, dann gilt:
    \begin{itemize}
        \item[1)] Für fast alle $x\in[a,b]$ ist $A_x\coloneqq\{y\in\mathbb{R}^n||\, (x,y)\in A\}\subseteq\mathbb{R}^n$ messbar und $\mu_{\mathbb{R}^n}(A_x) <\infty$
        \item[2)]$\mu_{\mathbb{R}^{n+1}}(A) = \integ{{[a,b]}}{{\mu_{\mathbb{R}^n}(A_x)}}$
    \end{itemize}
    Insebesondere: Gilt $\tilde{A} \subseteq [a,b]\times\mathbb{R}^n$ messbar mit $\mu_{\mathbb{R}^n}(\tilde{A}_x)=\mu_{\mathbb{R}^n}(A_x)$ für fast alle $x\in[a,b]$, dann gilt:
    \begin{align*}
        \mu_{\mathbb{R}^{n+1}}(\tilde{A})=\mu_{\mathbb{R}^n}(A)
    \end{align*}
\end{satz}

\underline{Beispiele:}\\
\begin{minipage}{0.333\textwidth}
    Abbildung Halbkugel
\end{minipage}
\begin{minipage}{0.666\textwidth}
Halbkugel mit Radius r.\\
Schnitt auf Höhe z: Kreis mir Radius $r_z, r^2 = z^2 +r_z^2$\\
Fläche $\pi r_z^2 = \pi(r^2-z^2)$
\end{minipage}

\begin{minipage}{0.333\textwidth}
    Abbildung Zylinder ohne Kegel
\end{minipage}
\begin{minipage}{0.666\textwidth}
    Schnitt auf Höhe z: Kreisring, Außenradius r, Innenradius z\\
    Fläche: $\pi r^2 - \pi z^2 = \pi(r^2-z^2)$\\
    \begin{align*}
        \Rightarrow \text{Volumen Halbkugel} = &\text{Volumen Restkörper}\\
        = & \text{ Volumen Zylinder} - \text{Volumen Kegel}\\
        = & \pi r^2 \cdot r - \frac{1}{3} \pi r^2\cdot r\\
        = & \frac{2}{3} \pi r^3
    \end{align*}
    $\Rightarrow \text{ Volumen Kugel } = \frac{4}{3} \pi r^3$
\end{minipage}
\\
\underline{Außblick:} (Maß- und Integrationstheorie)\\
Sei $\Omega = \mathbb{R}^n$. Die Menge $\mathscr{A} = \{A \subseteq \mathbb{R}^n | A \text{ messbar}\}$ ist eine $\sigma$-Algebra, das heißt $\emptyset \in \mathscr{A}$ und\\
$i) A_1, A_2, \dots \in \mathscr{A} \Rightarrow \bigcup_{i=1}^\infty A_i \in \mathscr{A}$\\
$ii) A\in\mathscr{A} \Rightarrow\Omega\setminus A\in\mathscr{A}$\\
Für das Borel-Lebesgue-Maß gilt:\\
$iii) \text{ Sind } A_1, A_2, \dots \in \mathscr{A}$ paarweise disjunkt, dann:
\begin{align*}
    \mu \left(\bigcup_{i=1}^\infty A_i \right) = \sum_{i=1}^\infty \mu (A_i) \text{ ($\mu$ ist $\sigma$-additiv)}
\end{align*}
Mit $\Omega_1 = \mathbb{R}^n, \Omega_2 = \mathbb{R}$ gilt: (mit $\mathscr{A}_1 \coloneqq \mathscr{A}, \mathscr{A}_2 \coloneqq \{A\subseteq\Omega_2| A \text{ messbar}\})$\\
$iv) f\colon \Omega_1\to\Omega_2$ ist messbar genau dann, wenn
\begin{align*}
    \forall A\in\mathscr{A}_2 \colon f^{-1}(A) \in \mathscr{A}_1
\end{align*}
$v)$ Eine messbare Funktion $f\colon\Omega\to\mathbb{R}$ ist integrierbar ($\Leftrightarrow f\in L(\Omega_1)$) g.d.w
\begin{align*}
    \integ{\Omega}{f(x)} \coloneqq lim_{\epsilon \rightharpoondown 0} \int (f_+, \epsilon) - lim_{\epsilon \rightharpoondown 0} \int (f_-, \epsilon)
\end{align*} im eigentlichen Sinn existiert.\\
wobei $f_+ \coloneqq \text{max}(f,0), f_-\coloneqq \text{max}(-f,0)$ und
\begin{align*}
    \int (f_+,\epsilon) \coloneqq \sum_{k=0}^\infty k\,\epsilon\,\mu (\{x\in\Omega |\,k\,\epsilon \leq f_{\pm}(x) < (k+1)\epsilon\})
\end{align*}
Für $\Omega=\mathbb{R}^n$ gilt dann $\integ{\Omega}{f(x)} = \integ{\mathbb{R}^n}{f(x)}$
\underline{Allgemein:}\\
Ein Tripel $(\Omega, \mathscr{A}, \mu)$ mit $\mathscr{A}\subseteq\mathscr{P}(\Omega) \sigma$-Algebra und $\mu\colon \mathscr{A}\to\mathbb{R}_{\geq 0} \cup \{+\infty\} \sigma$-additiv nennt man einen \underline{Maßraum}\\
Also: $\Omega = \mathbb{R}^n, \mathscr{A}=$ Menge aller messbaren Teilmengen von $\Omega=\mathbb{R}^n, \mu =$ Borell-Lebesgue-Maß bilden einen Maßraum.\\
Für allg. Maßräume ($\Omega, \mathscr{A},\mu$) sind die messbaren Funktionen durch $iv)$ definiert. Das Lebesgue-integral messbarer Funktionen $f\colon \Omega\to \mathbb{R}$ lässt sich mittels $v)$ definieren.\\

\underline{$L^P$-Räume}\\
Für $p>0: \mathscr{L}^P(\Omega, \mu)\coloneqq \{f\colon \Omega\to\mathbb{R} |\, |f|^P \in L(\Omega, \mu)\}$. Dass ist ein Vektorraum (o.Beweis)\\
Definition von Äquivalenzrelation ~ auf $\mathscr{L}^P(\Omega)\colon$ Seien $f,g\in\mathscr{L}^P(\Omega, \mu)$
\begin{align*}
    f~g\colon \Leftrightarrow f=g \text{ fast üebrall}
\end{align*}
$L^P(\Omega,\mu)\coloneqq$ Menge aller Äquivalenzklassen in $\mathscr{L}^P(\Omega, \mu)$\\
\underline{Dann:} $L^P(\Omega, \mu)$ ist (mit den kanonischen Verknüpfungen) ein Vektorraum
\begin{align*}
    ||f||_p\coloneqq \left(\integ{{\Omega}}{|f(x)|^p}\right)^{\frac{1}{p}} \text{ für } f\in L^P(\Omega,\mu)
\end{align*}
ist eine Norm auf $L^P(\Omega, \mu)$ d. h.
\begin{itemize}
    \item[i)] $\forall f\in L^P(\Omega,\mu)\colon ||f||_p \geq 0, \text{ mit } = 0 \text{g. d. w.} f= 0$
    \item[ii)] $\forall f\in L^P(\Omega,\mu), \alpha\in\mathbb{R}\colon ||\alpha f||_p = |\alpha| ||f||_p$
    \item[iii)] $\forall f,g\in L^P(\Omega,\mu)\colon ||f+g||_p \leq ||f||_p + ||g||_p$
\end{itemize}
\begin{satz}
    von Fischer-Riesz für $p=2$\\
    $L^2(\mathbb{R}^n)$ ist vollständig\\
    (d.h. jede Cauchyfolge in $L^2(\mathbb{R}^n)$ konvergiert, d.h. ist $f_1, f_2, f_3, \dots \in  L^2(\mathbb{R}^n)$ Cauchyfolge [d.h. $\forall_{\epsilon > 0} \exists_{N>0}\colon\forall_{n,m\geq N, \text{ mit }n \leq m} \colon ||f_m -f_n||<\epsilon]\\
    \exists f\in L^2(\mathbb{R}^n)\colon \forall_{\epsilon>0} \exists_{M>0}\colon \forall{n\geq M}\colon ||f_n-f||_p <\epsilon$)
\end{satz}

\begin{minipage}{0.333\textwidth}
    Zylinder
\end{minipage}
\begin{minipage}{0.666\textwidth}
    $D\subseteq\mathbb{R}^2, f\colon D\to\mathbb{R} \text{ stetig } (\Rightarrow f \in L(D))$ kompakt\\
    Volumen unter dem Graphen: = $\boxed{\int_D f(x,y) \, \text{d}(x,y) = \int_{\mathbb{R}^3} \id_{\tilde{D}}(x,y,z)\,\text{d}(x,y,z)}$ mit $\tilde{D}\coloneqq \{(x,y,z)\in\mathbb{R}^3 |\, (x,y)\in D, \,0 \leq z\leq f(x,y)\}\subseteq \mathbb{R}^3$ 
\end{minipage}
\\
Analog für andere $\mathbb{R}^n$, z. B. $n=1$:\\
$D\subseteq\mathbb{R}$ kompakt, $f\colon D\to\mathbb{R}_{\geq 0}$ stetig, $D=[a,b]$ und $f(x) < C \forall x\in D$\\
Fläche unter dem Graphen = $\integ{D}{f(x)} = \int_{\mathbb{R}^2} \id_{\tilde{D}}(x,y) \,\text{d}(x,y)$ mit $\tilde{D}\coloneqq\{(x,y)\in \mathbb{R}^2 |\, x\in D, \, 0\leq y\leq f(x)\}$
\begin{proof}
    $\tilde{D}$ ist Normalbereich $\Rightarrow$\begin{align*}
        \int_{\mathbb{R}^2} \id_{\tilde{D}}(x,y)\,\text{d}(x,y) = \int_a^b\left(\int_0^c \id_{\tilde{D}}(x,y)\,\text{d}y\right)\,\text{d}x = \int_a^b f(x) \,\text{d}x = \integ{D}{f(x)}
    \end{align*}
\end{proof}

\underline{Beispiele:}\\
\begin{itemize}
    \item \begin{minipage}{0.333\textwidth}
        Abbildung
    \end{minipage}
    \begin{minipage}{0.666\textwidth}
        $D=[0,1]\times[0,1] \subseteq\mathbb{R}^2, f\colon D\to \mathbb{R}, \, x\mapsto 1+x+2y$
    \end{minipage}
    Funktion ist stetig $\Rightarrow f\in L(D)$\\
     \begin{align*}
         \int_D f(x,y)\,\text{d}(x,y) = \int_0^1\left(\int_0^1 1+x+2y \,\text{d}x\right)\,\text{d}y\\
         = \int_0^1 \left[x+\frac{1}{2}x^2+2yx\right]_{x=0}^{x=1}\,\text{d}y\\
         = \int_0^1 1+\frac{1}{2}+2y\,\text{d}y\\
         = \left[y+\frac{1}{2}y+y^2\right]_{y=0}^{y=1}\\
         = 1+\frac{1}{2}+1=\frac{5}{2}
     \end{align*}
     \item \begin{minipage}{0.333\textwidth}
         Abbildung Halbkugel
     \end{minipage}
     \begin{minipage}{0.666\textwidth}
         $D=\{(x,y)\in\mathbb{R}^2 | \, x^2 +y^2 \leq r^2\} = \{(x,y)\in\mathbb{R}^2 | \, -r\leq x\leq r, -\sqrt{r^2 - x^2} \leq y\leq \sqrt{r^2 - x^2}\}$
         \begin{align*}
         f\colon D&\to\mathbb{R}_{\geq 0}\\
         (x,y)&\mapsto \sqrt{r^2 -x^2-y^2}\\
         \int_D f(x,y)\,\text{d}(x,y) &= \int_{-r}^r\left(\int_{-\sqrt{r^2 -x^2}}^{\sqrt{r^2 -x^2}} \sqrt{r^2-x^2-y^2}\,\text{d}y\right)\,\text{d}x\\
         &\vdots\\
         &= \frac{2}{3}\pi r^3
         \end{align*}
        \end{minipage}
\end{itemize}